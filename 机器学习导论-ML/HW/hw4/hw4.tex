\documentclass[a4paper,UTF8]{article}
\usepackage{ctex}
\usepackage[margin=1.25in]{geometry}
\usepackage{color}
\usepackage{graphicx}
\usepackage{amssymb}
\usepackage{amsmath}
\usepackage{amsthm}
\numberwithin{equation}{section}
\usepackage[ruled,vlined,lined,boxed,linesnumbered]{algorithm2e}
\usepackage{soul, color, xcolor}
\usepackage{bm}
\usepackage{tcolorbox}
\usepackage{hyperref}
\numberwithin{equation}{section}
%\usepackage[thmmarks, amsmath, thref]{ntheorem}
\theoremstyle{definition}
\newtheorem*{solution}{Solution}
\newtheorem*{prove}{Proof}
\usepackage{multirow}
\usepackage{diagbox}
\usepackage{float}
\usepackage{subfigure}

\setlength{\parindent}{0pt}
\newcommand{\bds}{\boldsymbol}
\def \X {\mathbf{X}}
\def \A {\mathbf{A}}
\def \w {\mathbf{w}}
\def \y {\mathbf{y}}
\def \x {\mathbf{x}}
\def \z {\mathbf{z}}
\def \hy {\widehat{y}}
\def \by {\Bar{y}}
\def \H {\mathbf{H}}
\def \I {\mathbf{I}}
\newcommand\given[1][]{\:#1\vert\:}
\def \Ent {\text{Ent}}

\begin{document}
\title{机器学习导论\ 习题四}
\author{211300063, 张运吉, \href{mailto:邮箱}{211300063@smail.nju.edu.cn}}
\maketitle
\section*{作业提交注意事项}
\begin{tcolorbox}
	\begin{enumerate}
		\item[1.] 请在LaTeX模板中第一页填写个人的学号、姓名、邮箱;
		\item[2.] 本次作业需提交作答后的该 pdf 文件、编程题~.ipynb 文件; {\color{red}\textbf{请将二者打包为~.zip 文件上传}}. 注意命名规则, 三个文件均命名为 “$\text{学号}\_\text{姓名}$” + “.后缀” (例如 $\text{211300001}\_\text{张三}$” + “.pdf”、“.ipynb”、“.zip”);
		\item[3.] 若多次提交作业, 则在命名~.zip 文件时加上版本号, 例如 $\text{211300001}\_\text{张三}$ $\_\text{v1.zip}$” (批改时以版本号最高的文件为准);
		\item[4.] 本次作业提交截止时间为 {\color{red}\textbf{ 5 月 24 日 23:59:59}}. 未按照要求提交作业, 提交作业格式不正确, {\color{red}\textbf{作业命名不规范}}, 将会被扣除部分作业分数; 除特殊原因 (如因病缓交, 需出示医院假条) 逾期未交作业, 本次作业记 0 分; {\color{red}\textbf{如发现抄袭, 抄袭和被抄袭双方成绩全部取消}};
		\item[5.] 本次作业提交地址为 \href{https://box.nju.edu.cn/u/d/90ad71e29ef240a49e9e/}{here}, 请大家预留时间提前上交, 以防在临近截止日期时, 因网络等原因无法按时提交作业.
	\end{enumerate}
\end{tcolorbox}
\newpage


\section{[15pts] Vanishing Gradient Problem}

在使用梯度下降与反向传播训练深度神经网络时, 可能会出现梯度消失的问题, 即网络参数的梯度非常小, 导致网络更新非常缓慢, 甚至停止更新.
该问题的成因较为复杂, 有很多因素可能导致该问题的出现. 本题将主要讨论激活函数与该问题之间的联系.

\begin{enumerate}
    \item[(1)] \textbf{[5pts]} 当在深度神经网络中采用 Sigmoid 激活函数时, 网络训练容易出现梯度消失问题. 为分析此现象, 请先求解 Sigmoid 导函数的值域, 并根据该范围, 进一步对该现象进行分析.
    \item[(2)] \textbf{[5pts]} 当前深度神经网络大多采用 ReLU 激活函数, 试分析相较于 Sigmoid, ReLU 对梯度消失问题的缓解作用, 同时思考其可能带来的一些问题.
    \item[(3)] \textbf{[5pts]} 请从激活函数之外的角度, 列举三项缓解梯度消失问题的措施. 
\end{enumerate}

\begin{solution}
    此处用于写解答(中英文均可)
	\begin{enumerate}
        \item [(1)]
        $\because\operatorname*{Sigmoid}^{\prime}(x) = \operatorname*{Sigmoid}(x)(1-\operatorname*{Sigmoid}(x)), \operatorname*{Sigmoid}(x) \in (0, 1)$.\\
        $\therefore \operatorname*{Sigmoid}^{\prime}(x) \in (0, 0.25)$ \\
        当神经网络的层数较多时,在反向传播过程中,每层都需要乘以导数来计算梯度。如果某些神经元的输出接近于饱和区间,即非常接近0 或 1,那么它们对应的导数将会非常小,很容易就会出现梯度消失的现象.
        \item [(2)]
        ReLU的缓解作用: \\
        ReLU 激活函数在正区间上导数始终为常数 1,因此不会出现梯度消失的情况,这是 ReLU 相较于 Sigmoid 的一个主要优势. \\
        潜在问题: \\
        a. 当神经元的输入小于零时,ReLU 函数的输出恒为零.如果输入一直为负数,那么后面所有的权重更新将全都失效.这会导致该神经元无法再次被激活,从而影响网络的表达能力. \\
        b. ReLU 的输出并没有做归一化处理,因此其输出分布不受限制,可能会出现某些神经元输出特别大或特别小的情况。如果训练过程中发生这种情况,可能会影响模型的泛化能力,并引起过拟合等问题.
        \item [(3)]
        除了使用特定的激活函数之外,还可以从以下角度尝试缓解梯度消失问题:
        \begin{itemize}
            \item a. 批标准化: \\批标准化指的是对神经网络每一层的输入进行标准化。在进行批标准化时,将每个输入样本的特征都归一化到均值为 0,方差为 1 的分布上。这有助于避免某些激活函数(如 sigmoid、tanh 等)中超过饱和区域产生的梯度消失。同时, 批标准化还可以作为一种正则化手段,帮助防止过拟合.
            \item b. 权重初始化: \\
            权重的过大或过小都容易导致梯度消失问题.因此,在初始化时通常采用一些较为中性的方式来初始化权重,使得输出分布在合适的范围内.如 xavier initialization、He initialization 等方法,会根据输入和输出的维度计算初始化权重的标准差,从而减少量纲的影响,使得输出数据的标准差更加平衡,加速模型收敛.
            \item c. 增加层数或者改变网络结构: \\
            增加网络层数可以扩展模型的表示能力,使得模型能够学习更加复杂的输入输出关系。如果网络层数不足,可能无法表达数据集中包含的潜在规律,因此也容易发生梯度消失问题. 然而,过多的层也会增加梯度消失的风险,而且还会出现训练变得更加困难、收敛速度减缓等问题. 因此,在实践中需要权衡深度和宽度,选择合适的网络结构来避免梯度消失问题。
        \end{itemize}
    \end{enumerate}
\end{solution}

\newpage

\section{[15pts] Derivation and Analysis of PCA}
记中心化样本$\mathbf{X}=\left(\boldsymbol{x}_1, \boldsymbol{x}_2, \ldots, \boldsymbol{x}_n\right) \in \mathbb{R}^{d \times n}$满足$\sum_i \boldsymbol{x}_i=\mathbf{0}$; 投影变换$\mathbf{W}=\left(\boldsymbol{w}_1, \boldsymbol{w}_2, \ldots, \boldsymbol{w}_d\right) \in \mathbb{R}^{d \times d}$, 每维是正交基向量, 满足$\left\|\boldsymbol{w}_i\right\|_2=1, \boldsymbol{w}_i^{\top} \boldsymbol{w}_j=0(\forall i \neq j)$.

\begin{enumerate}
    \item[(1)] \textbf{[5pts]} 用拉格朗日乘子法求解PCA的优化问题.
    \begin{align*}
        \underset{\mathbf{W}}{\max }\ \ &  \operatorname{tr}\left(\mathbf{W}^{\mathrm{T}} \mathbf{X} \mathbf{X}^{\mathrm{T}} \mathbf{W}\right)\\
        \text { s.t. }\ & \mathbf{W}^{\mathrm{T}} \mathbf{W}=\mathbf{I}
    \end{align*}
    
    \item[(2)] \textbf{[5pts]} 对于以下三个样本点: $\boldsymbol{x}_1=(-1, 1)^{\mathrm{T}}, \boldsymbol{x}_2=(0, -2)^{\mathrm{T}}, \boldsymbol{x}_3=(1, 1)^{\mathrm{T}}$, 试用(1)中得到的结果求解最大主成分对应的$\boldsymbol{w}_1$.
    
    \item[(3)] \textbf{[5pts]} 设原样本$\mathbf{X}$的协方差矩阵对应的$d$个特征值组成的投影变换为$\mathbf{W}$. 考虑以下三种变换: 平移(每个样本沿向量$\boldsymbol{q}$方向移动距离 $s$)、放缩(每个样本乘以放大率$\alpha$)和旋转(样本围绕点$\boldsymbol{p}$顺时针旋转$\theta$). 试求解变换后的样本$\hat{\mathbf{X}}$对应的$\hat{\mathbf{W}}$.

\end{enumerate}

\begin{solution}
    此处用于写解答(中英文均可)
    \begin{enumerate}
        \item [(1)]
        将原始问题写成优化问题的标准形式: 
        \begin{align*}
            \underset{\mathbf{W}}{\min }\ \ &  -\operatorname{tr}\left(\mathbf{W}^{\mathrm{T}} \mathbf{X} \mathbf{X}^{\mathrm{T}} \mathbf{W}\right)\\
            \text { s.t. }\ & \mathbf{W}^{\mathrm{T}} \mathbf{W} - \mathbf{I}= \boldsymbol{0}.
        \end{align*}
        此优化目标的拉格朗日函数为: 
        \begin{equation}
            \begin{aligned} L(\mathbf{W}, \Theta) & =-\operatorname{tr}\left(\mathbf{W}^{\mathrm{T}} \mathbf{X} \mathbf{X}^{\mathrm{T}} \mathbf{W}\right)+\left\langle\Theta, \mathbf{W}^{\mathrm{T}} \mathbf{W}-\mathbf{I}\right\rangle \\ & =-\operatorname{tr}\left(\mathbf{W}^{\mathrm{T}} \mathbf{X} \mathbf{X}^{\mathrm{T}} \mathbf{W}\right)+\operatorname{tr}\left(\Theta^{\mathrm{T}}\left(\mathbf{W}^{\mathrm{T}} \mathbf{W}-\mathbf{I}\right)\right)\nonumber\end{aligned}
        \end{equation}
        其中$\Theta \in \mathbb{R}^{d \times d}$, $\left\langle\Theta, \mathbf{W}^{\mathrm{T}} \mathbf{W}-\mathbf{I}\right\rangle=\operatorname{tr}\left(\Theta^{\mathrm{T}}\left(\mathbf{W}^{\mathrm{T}} \mathbf{W}-\mathbf{I}\right)\right)$为矩阵内积. \\
        若此时仅考虑约束:$\boldsymbol{w}_{i}^{\mathrm{T}} \boldsymbol{w}_{i}=1, i\in[d]$. 拉格朗日乘子矩阵$\Theta$此时为对角矩阵, 令 
        $$\Theta=\Lambda=\operatorname{diag}\left(\lambda_{1}, \lambda_{2}, \ldots, \lambda_{d}\right)$$
        可以如此处理的原因是此时$L$的第二项:
        $$\operatorname{tr}\left(\Theta^{\mathrm{T}}\left(\mathbf{W}^{\mathrm{T}} \mathbf{W}-\mathbf{I}\right)\right) = \sum\limits_{i=1}^d\lambda_{i}(\boldsymbol{w}_{i}^{\mathrm{T}} \boldsymbol{w}_{i}-1)$$
        拉格朗日函数变为:
        $$L(\mathbf{W}, \Lambda)=-\operatorname{tr}\left(\mathbf{W}^{\mathrm{T}} \mathbf{X} \mathbf{X}^{\mathrm{T}} \mathbf{W}\right)+\operatorname{tr}\left(\Lambda^{\mathrm{T}}\left(\mathbf{W}^{\mathrm{T}} \mathbf{W}-\mathbf{I}\right)\right)$$
        对其求导:
        \begin{align*} \frac{\partial L(\mathbf{W}, \Lambda)}{\partial \mathbf{W}} & =\frac{\partial}{\partial \mathbf{W}}\left[-\operatorname{tr}\left(\mathbf{W}^{\mathrm{T}} \mathbf{X} \mathbf{X}^{\mathrm{T}} \mathbf{W}\right)+\operatorname{tr}\left(\Lambda^{\mathrm{T}}\left(\mathbf{W}^{\mathrm{T}} \mathbf{W}-\mathbf{I}\right)\right)\right] \\ & =-\frac{\partial}{\partial \mathbf{W}} \operatorname{tr}\left(\mathbf{W}^{\mathrm{T}} \mathbf{X} \mathbf{X}^{\mathrm{T}} \mathbf{W}\right)+\frac{\partial}{\partial \mathbf{W}} \operatorname{tr}\left(\Lambda^{\mathrm{T}}\left(\mathbf{W}^{\mathrm{T}} \mathbf{W}-\mathbf{I}\right)\right)\end{align*}
        由矩阵微分公式:
        $$\frac{\partial}{\partial \mathbf{X}} \operatorname{tr}\left(\mathbf{X}^{\mathrm{T}} \mathbf{B} \mathbf{X}\right)=\mathbf{B} \mathbf{X}+\mathbf{B}^{\mathrm{T}} \mathbf{X}, \frac{\partial}{\partial \mathbf{X}} \operatorname{tr}\left(\mathbf{B X}^{\mathrm{T}} \mathbf{X}\right)=\mathbf{X B}^{\mathrm{T}}+\mathbf{X} \mathbf{B}$$
        得到:
        $$\begin{aligned} \frac{\partial L(\mathbf{W}, \Lambda)}{\partial \mathbf{W}} & =-2 \mathbf{X} \mathbf{X}^{\mathrm{T}} \mathbf{W}+\mathbf{W} \Lambda+\mathbf{W} \Lambda^{\mathrm{T}} \\ & =-2 \mathbf{X} \mathbf{X}^{\mathrm{T}} \mathbf{W}+\mathbf{W}\left(\Lambda+\Lambda^{\mathrm{T}}\right) \\ & =-2 \mathbf{X} \mathbf{X}^{\mathrm{T}} \mathbf{W}+2 \mathbf{W} \Lambda\end{aligned}$$
        令$\frac{\partial L(\mathbf{W}, \Lambda)}{\partial \mathbf{W}}=\mathbf{0}$,可得:
        $$\mathbf{X X}^{\mathrm{T}} \mathbf{W}=\mathbf{W} \Lambda$$
        展开可得:
        $$\mathbf{X X}^{\mathrm{T}} \boldsymbol{w}_{i}=\lambda_{i} \boldsymbol{w}_{i}, \quad i\in[d]$$
        上式为矩阵特征值和特征向量的定义,$\lambda_i, \boldsymbol{w}_i$表示矩阵$\mathbf{X X}^{\mathrm{T}}$的特征值和对应的单位特征向量. \\
        以上是仅考虑约束$\boldsymbol{w}_{i}^{\mathrm{T}} \boldsymbol{w}_{i}=1, i\in[d]$的结果,还需要考虑约束$\boldsymbol{w}_{i}^{\mathrm{T}} \boldsymbol{w}_{j}=0, i\neq j$. 因为$\mathbf{X X}^{\mathrm{T}}$是一个实对称矩阵,实对称矩阵的不同特征值所对应的特征向量之间相互正交,同一特征值的不同特征向量可以通过施密特正交化使其变得正交,所以通过上式求得的$\boldsymbol{w}_{i}$可以同时满足约束$\boldsymbol{w}_{i}^{\mathrm{T}} \boldsymbol{w}_{i}=1, \boldsymbol{w}_{i}^{\mathrm{T}} \boldsymbol{w}_{j}=0, i\neq j$. \\
        根据拉格朗日乘子法的原理可知,此时求得的结果仅是最优解的必要条件,
        $\mathbf{X X}^{\mathrm{T}}$有$d$个相互正交的单位特征向量,所以还需要从这$d$个特征向量里找出$d^{\prime}$个能使得目标函数达到最优值的特征向量作为最优解. \\
        将$\mathbf{X X}^{\mathrm{T}} \boldsymbol{w}_{i}=\lambda_{i} \boldsymbol{w}_{i}$代入目标函数,因为需要降到的维数为$d^{\prime}$,得:
        \begin{align*} \min _{\mathbf{W}}-\operatorname{tr}\left(\mathbf{W}^{\mathrm{T}} \mathbf{X} \mathbf{X}^{\mathrm{T}} \mathbf{W}\right) & =\max _{\mathbf{W}} \operatorname{tr}\left(\mathbf{W}^{\mathrm{T}} \mathbf{X X}^{\mathrm{T}} \mathbf{W}\right) \\ & =\max _{\mathbf{W}} \sum_{i=1}^{d^{\prime}} \boldsymbol{w}_{i}^{\mathrm{T}} \mathbf{X X}^{\mathrm{T}} \boldsymbol{w}_{i} \\ & =\max _{\mathbf{W}} \sum_{i=1}^{d^{\prime}} \boldsymbol{w}_{i}^{\mathrm{T}} \cdot \lambda_{i} \boldsymbol{w}_{i} \\ & =\max _{\mathbf{W}} \sum_{i=1}^{d^{\prime}} \lambda_{i} \boldsymbol{w}_{i}^{\mathrm{T}} \boldsymbol{w}_{i} \\ & =\max _{\mathbf{W}} \sum_{i=1}^{d^{\prime}} \lambda_{i}\end{align*}
        因此,只需要令$\lambda_1, \lambda_2, \ldots, \lambda_{d^{\prime}}$和 $\boldsymbol{w}_1, \boldsymbol{w}_2, \ldots, \boldsymbol{w}_{d^{\prime}}$ 分别为矩阵$\mathbf{X X}^{\mathrm{T}}$的前 $d^{\prime}$个最大的特征值和对应的单位特征向量就能得到最优解.
        \item [(2)]
        计算协方差矩阵: 
        \begin{align*}
            \mathbf{X}\mathbf{X}^\mathrm{T} = \begin{bmatrix} -1 & 0 &1  \\ 1 & -2 & 1\end{bmatrix} \cdot \begin{bmatrix} -1 & 1 \\0 & -2 \\1&1\end{bmatrix} = \begin{bmatrix}    2 & 0 \\ 0 & 6        \end{bmatrix}
        \end{align*}
        对应的特征多项式:
        \begin{align*}
            f(\lambda) = \vert \lambda \mathbf{I}-\mathbf{X}\mathbf{X}^\mathrm{T}\vert = \begin{vmatrix} \lambda-2 & 0 \\ 0 & \lambda-6\end{vmatrix} = 0
        \end{align*} 
        解得: $\lambda = 2$ 或 $\lambda = 6$. \\
        对应的特征向量分别为 $\boldsymbol{y}_1 = (1, 0)^\mathrm{T}, \boldsymbol{y}_2=(0,1)^\mathrm{T}$. 由(1)中结论可知最大主成分对应的 $\boldsymbol{w}_1 = (0, 1)^\mathrm{T}$
        \item [(3)]
        \begin{enumerate}
            \item [a. ] 平移: \\
            每个样本沿$\boldsymbol{q}$方向平移$s$, 则有:
            $$\hat{\mathbf{X}}=\mathbf{X}+s \mathbf{q} \mathbf{1}^{T}$$
            其中$\boldsymbol{1}$是维数为n的全1列向量. \\
            将平移后的样本中心化后,求协方差矩阵: 
            $$\hat{\mathbf{H}}=(\hat{\mathbf{X}}-s \mathbf{q} \mathbf{1}^{T})(\hat{\mathbf{X}}-s \mathbf{q} \mathbf{1}^{T})^\mathrm{T}=\mathbf{X}\mathbf{X}^\mathrm{T}$$
            所以投影矩阵不变: $$\hat{\mathbf{W}} = \mathbf{W}$$
            \item [b. ] 放缩: \\
            每个样本乘以缩放率$\alpha$,则有: 
            $$\hat{\mathbf{X}}=\alpha \mathbf{X}$$
            由于是数乘变换,不需要对变化后的矩阵再进行中心化操作.\\
            协方差矩阵$$\hat{\mathbf{H}}=\alpha^2 \mathbf{X}\mathbf{X}^\mathrm{T}$$数乘后矩阵的单位特征向量不变,因此放缩变换之后 $\hat{\mathbf{W}}=\mathbf{W}$.
            \item [c. ] 旋转: \\
            每个样本围绕点$\boldsymbol{p}$顺时针旋转$\theta$,可以看成把样本点平移到坐标原点进行旋转,最后把样本点再反向平移,则有:
            $$\hat{\mathbf{X}}=\mathbf{R}(\theta)(\mathbf{X}-\boldsymbol{p}\ \cdot \ \mathbf{1}^\mathrm{T}) + \boldsymbol{p}\ \cdot \ \mathbf{1}^\mathrm{T}$$
            由于平移变换不改变投影矩阵,所以我们不妨考虑围绕坐标原点的旋转:
            $$\hat{\mathbf{X}}=\mathbf{R}(\theta)\mathbf{X}$$
            计算协方差矩阵:
            $$\hat{\mathbf{H}}=\mathbf{R}(\theta)\mathbf{X}\mathbf{X}^\mathrm{T}\mathbf{R}(\theta)^\mathrm{T}$$
            由于旋转矩阵均为正交矩阵,即 $\mathbf{R}(\theta)^{-1}=\mathbf{R}(\theta)^\mathrm{T}$,令 $\mathbf{P}=\mathbf{R}(\theta)^\mathrm{T}$,得到 $$\hat{\mathbf{H}}=\mathbf{P}^{-1}\mathbf{X}\mathbf{X}^\mathrm{T}\mathbf{P}$$根据特征向量的性质,可以得到若 $\boldsymbol{w}$为原协方差矩阵$\mathbf{X X}^{\mathrm{T}}$的一个特征向量, $\hat{\boldsymbol{w}}=\mathbf{R}(\theta)\boldsymbol{w}$为新协方差矩阵$\hat{\mathbf{H}}$的特征向量. 所以在旋转变换后 $$\hat{\mathbf{W}}=\mathbf{R}(\theta)\mathbf{W}$$.
        \end{enumerate}
    \end{enumerate}
\end{solution}

\newpage



\section{[35pts] Theoretical Analysis of $k$-means Algorithm}
给定样本集$\mathcal{D} = \{ \mathbf x_1,\mathbf x_2, \ldots, \mathbf x_n \}$, $k$-means聚类算法希望获得簇划分$\mathcal{C}=\{C_1,C_2,\cdots,C_k\}$, 使得最小化欧氏距离
\begin{equation}
\label{eq-kmeans-l2}
J(\gamma, \mu_1,\ldots,\mu_k) = \sum_{i=1}^n \sum_{j=1}^k \gamma_{ij}\|\mathbf x_i - \mu_j\|^2,
\end{equation}

其中 $\mu_1, \ldots, \mu_k$为$k$个簇的中心(means), $\gamma \in \mathbb{R}^{n\times k}$为指示矩阵(indicator matrix). $\gamma$ 具体定义如下:若$\mathbf x_i$属于第$j$个簇, 则$\gamma_{ij} = 1$, 否则为0.
算法~\ref{algo:kmeans} 中所示为经典 $k$-means 聚类算法的具体流程 (与课本中描述稍有差别, 但实际上是等价的).

\begin{algorithm}[]
\label{algo:kmeans}
\caption{$k$-means Algorithm}
\setcounter{AlgoLine}{0}
Initialize $\mu_1, \ldots, \mu_k$.\\
\Repeat{the objective function $J$ no longer changes}{
\textbf{Step 1}: Decide the class memberships of $\{\mathbf x_i\}_{i=1}^n$ by assigning each of them to its nearest cluster center.
\begin{align*}
\gamma_{ij} =
\begin{cases} 
1,& \|\mathbf x_i - \mu_j\|^2 \le \|\mathbf x_i - \mu_{j'}\|^2, \forall j', \\
0, & \text{otherwise}. 
\end{cases}
\end{align*}\\
\textbf{Step 2}: For each $j \in \{1, \cdots, k\}$, recompute $\mu_j$ using the updated $\gamma$ to be the center of mass of all points in $C_j$: 
\begin{align*}
\mu_j = \frac{\sum_{i=1}^n \gamma_{ij}\mathbf x_i}{\sum_{i=1}^n \gamma_{ij}}.
\end{align*}
}
\end{algorithm}

\begin{enumerate}

\item[(1)] \textbf{[5pts]} 试证明, 在算法~\ref{algo:kmeans} 中, \textbf{Step 1} 和 \textbf{Step 2} 都会使目标函数 $J$ 的值降低.

\item[(2)] \textbf{[5pts]} 试证明, 算法~\ref{algo:kmeans} 会在有限步内停止.

\item[(3)] \textbf{[5pts]} 试证明, 目标函数 $J$ 的最小值是关于 $k$ 的非增函数, 其中 $k$ 是聚类簇的数目.

\item[(4)] \textbf{[10pts]} 记 $\hat{\mathbf{x}}$ 为 $n$ 个样本的中心点, 定义如下变量:
\begin{table}[h]
\centering
\label{table:equation}
\begin{tabular}{ l | c }
  \hline			
total deviation & $T(X) = \sum_{i=1}^n \lVert \mathbf x_i - \hat{\mathbf x}\rVert^2/n$ \\
intra-cluster deviation & $W_j(X) = \sum_{i=1}^n \gamma_{ij} \lVert\mathbf x_i - \mu_j \rVert^2/\sum_{i=1}^n \gamma_{ij}$ \\
inter-cluster deviation & $B(X) = \sum_{j=1}^k \frac{ \sum_{i=1}^n \gamma_{ij}}{n}  \lVert\mu_j -\hat{\mathbf x} \rVert^2$\\
  \hline  
\end{tabular}
\end{table}

试探究以上三个变量之间有什么样的等式关系? 基于此, 请证明, $k$-means 聚类算法可以认为是在最小化 intra-cluster deviation 的加权平均, 同时近似最大化 inter-cluster deviation.

\item[(5)] \textbf{[10pts]} 在公式~\ref{eq-kmeans-l2} 中, 我们使用$\ell_2$-范数来度量距离(即欧氏距离), 下面我们考虑使用$\ell_1$-范数来度量距离:

\begin{equation}
\label{eq-kmeans-l1}
J'(\gamma, \mu_1,\ldots,\mu_k) = \sum_{i=1}^n \sum_{j=1}^k \gamma_{ij}\|\mathbf x_i - \mu_j\|_1.
\end{equation}

\begin{itemize} 
\item \textbf{[5pts]} 请仿效算法~\ref{algo:kmeans} ($k$-means-$\ell_2$算法), 给出新的算法(命名为$k$-means-$\ell_1$算法)以优化公式~\ref{eq-kmeans-l1} 中的目标函数$J'$.
\item \textbf{[5pts]} 当样本集中存在少量异常点(\href{https://en.wikipedia.org/wiki/Outlier}{outliers})时, 对于上述的$k$-means-$\ell_2$和$k$-means-$\ell_1$算法, 我们应该如何选择? 请从算法鲁棒性的角度分析, 说明哪个算法具有更好的鲁棒性?
\end{itemize}

\end{enumerate}

\begin{solution}
    此处用于写解答(中英文均可)
    \begin{enumerate}
        \item [(1)]
        \begin{itemize}
            \item Step1 \\
            假设step1更新前样本$x_l$属于簇$C_j$,更新后属于簇$C_{j^{\prime}}$,那么由更新条件有: $$\|\mathbf{x}_l-\mu_{j^\prime}\|^2\leq \|\mathbf{x}_p-\mu_{j}\|^2$$
            记经过step1所有发生簇变换的样本的下标集合$\mathcal{L}$. \\
            则:
            \begin{align*}
                J &= \sum\limits_{l\in \mathcal{L}}\|\mathbf{x}_{l}-\mu_{j_{l}}\|^2 + \sum_{i\not\in \mathcal{L}}\sum_{j=1}^k\gamma_{ij}\|\mathbf{x}_i-\mu_j\|^2  \quad \\
                J' &= \sum\limits_{l\in \mathcal{L}}\|\mathbf{x}_{l}-\mu_{j^{\prime}_{l}}\|^2 + \sum_{i\not\in \mathcal{L}}\sum_{j=1}^k\gamma_{ij}\|\mathbf{x}_i-\mu_j\|^2     \quad 
            \end{align*}
            由于$\sum\limits_{l\in \mathcal{L}}\|\mathbf{x}_{l}-\mu_{j^{\prime}_{l}}\|^2 \leq \sum\limits_{l\in \mathcal{L}}\|\mathbf{x}_{l}-\mu_{j_{l}}\|^2$,所以$J^{\prime} \leq J$.
            \item Step2 \\
            Step2步骤是在更新Step1得到的簇的中心点,欲证明在更新了中心点之后,$J^{\prime}\leq J$,只需证明新的中心点是使得新簇中所有点的中心距最小的点,即:
            \begin{align*}
                \mu_j = \arg \min_{\mu}\sum_{i=1}^n \gamma_{ij}\|\mathbf{x}_i-\mu\|^2, \quad \text{ 其中 } \mu_j \text{ 是更新后的中心点.}
            \end{align*} 
            目标函数对$\mu$求导并令其等于0,可得:
            \begin{align*}
                \frac{\partial}{\partial \mu} \sum_{i=1}^n \gamma_{ij}\|\mathbf x_i - \mu_j\|^2 = -2 \sum_{i=1}^n \gamma_{ij} (\mathbf x_i - \mu_j) = 0
            \end{align*} 
            解之得:
            \begin{align*}
                \mu_j = \frac{\sum_{i=1}^n \gamma_{ij}\mathbf x_i}{\sum_{i=1}^n \gamma_{ij}}
            \end{align*}
            正好是Step2对应得更新规则. 由此可得$J^{\prime}\leq J$.
        \end{itemize}
        \item [(2)]
        算法1是一个迭代的过程,每次迭代目标函数 $J$都会单调递减,且$J$有下界0. 因此我们只需要证明,在有限次迭代之后,算法会终止. \\
        假设在第t次迭代后,经过Step1和Step2步骤得到的新的簇中心与之前的相同,即$\mu_i^{(t)} = \mu_i^{(t-1)}, i\in [k]$. 根据算法的定义, 对于任意数据点$x_i$, 它属于原先的某个簇$C_j^{(t-1)}$并且$\|x_i-\mu_j^{(t-1)}\| \leq \|x_i-\mu_{j^\prime}^{(t-1)}\|$对所有$j \neq j^{\prime}$都成立. 而由于$\mu_i^{(t)} = \mu_i^{(t-1)}$, 所以上述不等式仍然成立. 因此,在第t次迭代后,所有的数据点都属于原先的簇,并没有发生改变,所以此时目标函数$J$也没有发生改变,算法会终止. \\
        综上所述, 算法~\ref{algo:kmeans} 会在有限步内停止.
        \item [(3)]
        假设当前得到一个最优簇划分$\mathcal{C}=\{C_1,C_2,\cdots,C_k\}$,此时目标函数$J(k)$达到最小$J(k)_{min}$. 若此时新增加一个簇$C_{k+1}$, 初始状态下$C_{k+1}$中仅含一个样本点(这个样本点可以任取一个), 则$\sum_{i=1}^{n} \gamma_{i,k+1}\left\|\mathbf{x}_{i}-\mu_{k+1}\right\|^{2}=0$, 记此时的目标函数为$J(k+1)$, $J(k+1)=\sum_{i=1}^n \sum_{j=1}^{k+1} \gamma_{ij}\|\mathbf x_i - \mu_j\|^2=J(k)+\sum_{i=1}^{n} \gamma_{i,k+1}\left\|\mathbf{x}_{i}-\mu_{k+1}\right\|^{2}$,有$J(k+1) = J(k)_{min}$. 由(1)中结论,每一次迭代都会使$J(k+1)$的值变小,因此最终有$J(k+1)_{min} \leq J(k)_{min}$. 即目标函数 $J$ 的最小值是关于 $k$ 的非增函数. 
        \item [(4)]
        \begin{align*} T(X) & =\frac{\sum_{i=1}^{n}\left\|\mathbf{x}_{i}-\hat{\mathbf{x}}\right\|^{2}}{n} \\ &= \frac{\sum_{i=1}^{n}\left\|\mathbf{x}_{i}- \mu_{x_i} + \mu_{x_i}-\hat{\mathbf{x}}\right\|^{2}}{n}\quad \text{ ($\mu_{x_i}$表示$x_i$所在簇的中心)} \\ &= \frac{\sum_{i=1}^{n}\sum_{j=1}^{k}\gamma_{ij}\left\|\mathbf{x}_{i}- \mu_{j} + \mu_{j}-\hat{\mathbf{x}}\right\|^{2}}{n}\\& =\frac{1}{n} \sum_{i=1}^{n}\sum_{j=1}^{k}\gamma_{ij}\left\|\mathbf{x}_{i}-\mu_{j}\right\|^{2}+\frac{1}{n} \sum_{i=1}^{n}\sum_{j=1}^{k}\gamma_{ij}\left\|\mu_{j}-\hat{\mathbf{x}}\right\|^{2}+ \frac{2}{n}\sum_{i=1}^{n}\sum_{j=1}^{k}\gamma_{ij}\left(\mathbf{x}_{i}-\mu_{j}\right)^{\mathrm{T}}\left(\mu_{j}-\hat{\mathbf{x}}\right) \\
        W(X) &=\frac{1}{n}\sum_{j=1}^{k}\left(W_j(X)\cdot\sum_{i=1}^n\gamma_{ij}\right)=\frac{1}{n}\sum_{j=1}^k \sum_{i=1}^n \gamma_{ij} \lVert\mathbf x_i - \mu_j \rVert^2 \\
        B(X) &= \sum_{j=1}^k \frac{ \sum_{i=1}^n \gamma_{ij}}{n}  \lVert\mu_j -\hat{\mathbf x} \rVert^2 =\frac{1}{n} \sum_{i=1}^{n}\sum_{j=1}^{k}\gamma_{ij}\left\|\mu_{j}-\hat{\mathbf{x}}\right\|^{2}\\
        \end{align*}
        而: $$\frac{2}{n}\sum_{i=1}^{n}\sum_{j=1}^{k}\gamma_{ij}\left(\mathbf{x}_{i}-\mu_{j}\right)^{\mathrm{T}}\left(\mu_{j}-\hat{\mathbf{x}}\right) = \frac{2}{n}\sum_{j=1}^{k}\left[\sum_{i=1}^{n}\gamma_{ij}\left(\mathbf{x}_{i}-\mu_{j}\right)^{\mathrm{T}}\right]\left(\mu_{j}-\hat{\mathbf{x}}\right)$$
        对于某个确定的簇$C_j$, $\sum_{i=1}^{n}\gamma_{ij}\left(\mathbf{x}_{i}-\mu_{j}\right)^{\mathrm{T}} = 0$, 因为该簇的所有点到簇中心的向量之和是零向量.
        所以我们有:
        \begin{align*}
            T(X) &=  W(X) + B(X) 
        \end{align*}
        即$T(X)$等于intra-cluster deviation 的加权平均加上 inter-cluster
        deviation. \\
        对于给定的样本$T(X)$是定值,而$W(X)=\frac{J}{n}$, $J$是目标函数.\\
        因此最小化$J$的过程相当于最小化$W(X)$,同时近似最大化$B(X)$, 也即k-means 聚类算法可以认为是在最小化 intra-cluster deviation 的加权平均, 同时近似最大化 inter-cluster
        deviation.
        \item [(5)]
        \begin{itemize}
            \item $k$-means-$\ell_1$算法
            \begin{algorithm}[]
                \label{algo:kmeans-1}
                \caption{$k$-means-$\ell_1$  Algorithm}
                \setcounter{AlgoLine}{0}
                Initialize $\mu_1, \ldots, \mu_k$.\\
                \Repeat{the objective function $J$ no longer changes}{
                \textbf{Step 1}: Decide the class memberships of $\{\mathbf x_i\}_{i=1}^n$ by assigning each of them to its nearest cluster center.
                \begin{align*}
                \gamma_{ij} =
                \begin{cases} 
                1,& \|\mathbf x_i - \mu_j\|_1 \le \|\mathbf x_i - \mu_{j'}\|_1, \forall j', \\
                0, & \text{otherwise}. 
                \end{cases}
                \end{align*}\\
                \textbf{Step 2}: For each $j \in \{1, \cdots, k\}$, recompute $\mu_j$ using the updated $\gamma$ to be the median of all points in $C_j$: 
                \begin{align*}
                \mu_j = \text{Median}\{\mathbf x_i\ | \ \gamma{ij} = 1\}.
                \end{align*}
                }
            \end{algorithm}
            \item 从算法鲁棒性的角度考虑,在存在异常点的情况下,$k$-means-$\ell_1$算法比$k$-means-$\ell_2$算法算法具有更好的鲁棒性. \\
            首先,因为$k$-means-$\ell_2$算法的距离度量是$\ell_2$范数的平方,如果有异常点,那么计算出来的$\ell_2$范数距离会比较大,  $k$-means-$\ell_1$使用曼哈顿距离,尽管存在异常点,但据此计算出来的距离不会比$\ell_2$范数距离大.\\
            其次, $k$-means-$\ell_2$算法簇中心点是求所有在该簇中的点的均值,如果有异常点,异常点对中心点的计算影响比较大,使得计算出来的中心点偏离实际中心点较远,而
            $k$-means-$\ell_1$算法则是取中位数,少量异常点对中位数的影响比较小,从而计算出来的中心点相对实际中心点的偏离不会太大.
        \end{itemize}
    \end{enumerate}
\end{solution}

\newpage

\section{[35pts] Neural Network in Practice}

本题需编程实现多层前馈神经网络, 且不使用现有神经网络库, 具体内容见 lab.ipynb 文件.

\subsection{任务要求}

在不使用现有神经网络库的前提下, 复现 lab.ipynb 中利用 PyTorch 编写的一段多层前馈神经网络代码, 需注意:
\begin{itemize}
    \item 保持网络结构一致;
    \item 保持损失函数一致;
    \item 保持 batch\_size, learning\_rate, epochs 超参一致.
\end{itemize}
编写时可参考 lab.ipynb 文件中给出的代码框架, 也可以将其删除, 编写你自己的代码. 任务完成后, 需提交对应的~.ipynb 文件.

\subsection{评分标准}

我们会重新运行你所提交的~.ipynb 文件, 具体评分标准如下:
\begin{itemize}
    \item 5/35: 复现的代码可正常运行;
    \item 10/35: 复现的代码保持了网络结构、损失函数以及各超参一致;
    \item 15/35: 复现的代码可在 5min 内迭代完规定的 epochs 轮数;
    \item 25/35: 运行结果中 running\_loss 整体为下降趋势;
    \item 35/35: 运行结果中最后一轮的 running\_acc 超过 90\%.
\end{itemize}

\subsection{测试环境}

以下为测试环境采用的版本:
\begin{itemize}
    \item Python: 3.8.16
    \item PyTorch: 2.0.0
    \item Numpy: 1.24.2
\end{itemize}
你可以使用你喜欢的任意版本, 只需确保你的代码能顺利运行.



\end{document}
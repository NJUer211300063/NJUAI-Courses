% 导言区
\documentclass{article}
\usepackage{ctex}  % 添加中文包
\usepackage{multirow} % 表格单元占据多个行或者列
\usepackage{listings} % 插入代码
\usepackage{xcolor}
\lstset{    % 设置代码样式
	numbers=left, 
	numberstyle= \tiny, 
	keywordstyle= \color{ blue!70},
	commentstyle= \color{red!50!green!50!blue!50}, 
	frame=shadowbox, % 阴影效果
	rulesepcolor= \color{ red!20!green!20!blue!20} ,
	escapeinside=``, % 英文分号中可写入中文
	xleftmargin=2em,xrightmargin=2em, aboveskip=1em,
	framexleftmargin=2em
} 
\usepackage{amssymb} % 和数学公式有关的包
\title{ICS homework3} 
\author{张运吉 211300063} 
\date{3/11/2022}

% 正文区 
\begin{document} 
\maketitle % 必须有这个才能显示标题等信息
\section{P6}	
(1).$R[ebp]+8, R[ebp]+12, R[ebp]+16$

(2).
\lstset{language=C}
\begin{lstlisting}
void func(int *xptr, int *yptr, int *zptr) {
	int x = *xptr;
	int y = *yptr;
	int z = *zptr;
	*yptr = x;
	*zptr = y;
	*xptr = z;		

}
\end{lstlisting}
\section{P8}
(1). 指令功能:$R[edx] \gets R[edx]+M[R[eax]]$

	 即$R[edx] \gets 0xfffffff0+0x80$
	 
	 所以EDX寄存器的值变为$0x70$	
	 
	 $OF=0,ZF=0,SF=0,CF=1$
	 
(2). 指令功能:$R[ecx] \gets R[ecx]-M[R[eax]+R[ebx]]$

即$R[ecx] \gets 0x10-0x80000008 = 0x10+0x7ffffff8=0x80000008$

所以ECX寄存器的值变为$0x80000008$	

$OF=1,ZF=0,SF=1,CF=1$

(3). 指令功能:$R[bx] \gets R[bx]+M[R[eax]+8*R[ecx]+4]$

即$R[bx] \gets 0x0100|0xff00=0xff00$

所以BX寄存器的值变为$0xff00$	

$OF=0,ZF=0,SF=1,CF=0$

(4). 指令功能:根据运算结果改变条件寄存器中的条件标记

$\because 0x80 \& R[dl]=0x80 \& 0x80=0x80$

$\therefore OF=0,ZF=0,SF=1,CF=0$

(5). 指令功能:$R[ecx] \gets M[R[eax]+R[edx]] * 32$

即$R[ecx] \gets 0x908f12a8*32 = 0x11e25500$

所以ECX寄存器的值变为$0x11e25500$	

$OF=1,ZF=0,SF=1,CF=1$

(6). 指令功能:$R[dx-ax] \gets R[ax]*R[bx]$

即$R[dx-ax] \gets 0x9300*0x0100 = 0x00930000$

所以DX寄存器的值变为$0x0093$,AX寄存器的值变为$0x0000$

$OF=CF=1$

(7). 指令功能:$R[cx] \gets R[cx]-1$

即$R[cx] \gets 0x0010-1 = 0x0010+0xffff=0x000f$

所以CX寄存器的值变为$0x000f$	

$OF=0,ZF=0,SF=0$

\section{P9}
1.$movl \quad 12(\% ebp), \% ecx$  \quad //$R[ecx] \gets M[R[ebp]+12]$, y送ECX

2.$sall \quad \$ 8, \% ecx$  \quad //$R[ecx] \gets R[ecx]<<8$, y*256送ECX

3.$movl \quad 8(\% ebp), \% eax$  \quad //$R[eax] \gets M[R[ebp]+8]$ ,x送EAX

4.$movl \quad 20(\% ebp), \% edx$ \quad  //$R[edx] \gets M[R[ebp]+20]$, k送EDX

5.$imull \quad \% edx, \% eax$  \quad //$R[eax] \gets R[eax]+R[edx]$, kx送EAX

6.$movl \quad 16(\% ebp), \% edx$  \quad //$R[edx] \gets M[R[ebp]+16]$ ,z送EDX

7.$andl \quad \$ 65520, \% edx$  \quad //$R[edx] \gets R[edx] \& 65520$, z \& 0xfff0 送EDX

8.$andl \quad \% ecx, \% edx$  \quad //$R[edx] \gets R[edx] \& R[ecx]$

9.$subl \quad \% edx, \% eax$  \quad //$R[eax] \gets R[eax] - R[edx] $

$\therefore$第3行:$int \quad v = k * x - (y * 256 + z \& 0xfff0)$

\section{P11}
(1) 目标地址:$0x804838c+2+0x08=0x8048396$

执行call指令时,$(PC)=0x804838e+5=0x8048393$

$\because 0x80483b1=0x8048393+0x1e$

$\therefore$可以得出call指令后4个字节是偏移量,第一个字节是操作码

转移目标地址$=(PC)+$偏移量

(4) 目标地址:$0x804829b+0xffffff00=0x804819b$

\section{P14}
(1)

1.$movw \quad 8(\% ebp), \% bx$  \quad //$R[bx] \gets M[R[ebp]+8]$, x送BX

2.$movw \quad 12(\% ebp), \% si$  \quad //$R[si] \gets M[R[ebp]+12]$, y送AX

3.$movw \quad 16(\% ebp), \% cx$  \quad //$R[cx] \gets M[R[ebp]+16]$, k送CX

4. $.L1:$

5.$movw \quad \% si, \% dx$  \quad //$R[dx] \gets R[si]$, y送DX

6.$movw \quad \% dx, \% ax$  \quad //$R[ax] \gets R[dx]$, y送AX

7.$sarw \quad \$ 15, \% dx$  \quad //$R[dx] \gets R[dx] >> 15$, DX中存y的符号

8.$idiv \quad \% cx$  \quad //$R[ax] \gets R[dx-ax] \div R[cx]$,  $y \div k$ 的商送AX

\qquad \qquad \qquad //$R[dx] \gets R[dx-ax] \% R[cx]$,  $y \div k$ 的余数送CX

9.$imulw \quad \% dx, \% bx$  \quad //$R[bx] \gets R[bx] - R[dx] $, x*(y\%k)送BX

10.$decw \quad \% cx$  \quad //$R[cx] \gets R[cx]-1$,  k-1送CX

11.$testw \quad \% cx, \%cx$  \quad //$R[cx] \& R[cx]$,  OF=CF=0

12.$jle \quad .L2$ \quad // 若k小于等于0,则装.L2

13.$cmpw \quad \% cx, \% si$  \quad //$R[si] - R[cx]$, $y-k$

14.$jg \quad .L1$ \quad // 若$R[si] > R[cx]$($y>k$)转L2

15.$.L2$

16.$movswl \quad \% bx, \% eax$  \quad //$R[eax] \gets R[bx]$, x*(y\%k)送AX

(2)被调用者保存寄存器:BX,SI

调用者保存寄存器:AX, CX, DX

EBX和ESI必须保存到栈中

(3)使DX中保存y的符号,这样在第8行的除法操作中R[dx-ax]保存的就是y符号拓展后的机器数。

\section{P16}
对应关系:

$x+3=0/x=-3 \quad .L7$

$x+3=1/x=-2 \quad .L2$

$x+3=2/x=-1 \quad .L2$

$x+3=3/x=0 \quad .L3$

$x+3=4/x=1 \quad .L4$

$x+3=5/x=2 \quad .L5$

$x+3=6/x=3 \quad .L6$

$x+3=7/x=4 \quad .L6$

switch-case语句原型:

\begin{lstlisting}
	switch(x) {
		case -2:case -1:
			... // 标号.L2
			break;
		case 0:
			... // 标号.L3
			break;
		case 1:
			... // 标号.L4
			break;
		case 2:
			... // 标号.L5
			break;
		case 4:
			... // 标号.L6
			break;
		default:
			... // 标号.L7
	}

\end{lstlisting}

由此可知,当x不属于前6种case时会执行default分支,标号为-2和-1会执行同一个case分支
\section{P17}
由第2、3行可知:a时char型,因为movsbw是字符串处理指令,且是移动一个字节,p是short *型。

由第4、5行可知:b、c是unsigned short型,因为movzwl是从2字节零拓展到4字节并移动。

由第6行可知返回类型是unsigned int。

综上,test的原型为:

$unsigned \quad int \quad test(char \quad a, unsigned \quad short \quad b, unsigned \quad short \quad c, short \quad *b)$

\section{P18}
\subsection{18.1}
$R[ebp]=0xbc000020-0x4=0xbc00001c$

$R[ebp]=0xbc00001c$

$R[ebp]=0xbc000030$

\subsection{18.2}
$R[esp]=r[ebp]=0xbc00001c$

$R[esp]=0xbc00001c-40-4=0xbbfffff0$

$R[esp]=0xbc00001c+4=0xbc000020$

\subsection{18.3}
x的地址: $0xbc00001c-4=0xbc000018$

y的地址: $0xbc00001c-8=0xbc000014$

\newpage
\subsection{18.4}
\begin{table}[h!]
	\begin{center}
	  \begin{tabular}{|c|} % <-- Alignments: 1st column left, 2nd middle and 3rd right, with vertical lines in between
		\hline
		0xbc000030 \\
		\hline
		$x=15$\\
		\hline
		$y=20$\\
		\hline
		...\\
		\hline
		...\\
		\hline
		...\\
		\hline
		0xbc000014\\
		\hline
		0xbc000018\\
		\hline
		0x804c0000\\
		\hline
		从scanf的返回地址\\
		\hline
		...\\
		\hline
		...\\
		\hline
		...\\
	  \end{tabular}
	\end{center}
\end{table}
最顶层一格地址是0xbc00001c,往下每走一格减4,格子“从scanf的返回地址”地址是0xbbfffff0
\section{P23}
执行第11行指令后,a[i][j][k]的地址是$a+4*(63i+9j+k)$

$\therefore MN=63, N=9 \Rightarrow M=7$

由第12行知$sizeof(a)=4536$

$\therefore LMN= \frac{4536}{64}=1134$

$\therefore L=\frac{1134}{63}=18$

综上,$L=18,M=7,N=9$

\newpage
\section{P26}
\begin{table}[h]
	\begin{center} % 表格居中
	  \begin{tabular}{c|c|c} % <-- Alignments: 1st column left, 2nd middle and 3rd right, with vertical lines in between
		\textbf{表达式EXPR} & \textbf{TYPE类型} & \textbf{汇编指令序列}\\
		\hline
		\multirow{2}{*}{uptr->s1.y}  & \multirow{2}{*}{short} & $movw \quad 4(\% eax), \% ax$\\ 
		&&$movw \quad \% ax, (\% edx)$\\
		\hline
		\multirow{2}{*}{\&uptr->s1.z} & \multirow{2}{*}{short *} & $lead \quad 6(\%eax), \%eax$\\
		&&$movw \quad \%eax, (\% edx)$\\
		\hline
		uptr->s2.a & short * & $movw \quad \%eax, (\% edx)$\\
		\hline
		\multirow{3}{*}{uptr->s2.a[uptr->s2.b]} &\multirow{3}{*}{short} & $movl \quad 4(\% eax), \% ecx$\\
		&&$movl \quad (\% eax, \% ecx, 2), \% eax$\\
		&&$movw \quad \%eax, (\%edx)$\\
		\hline
		\multirow{3}{*}{uptr->s2.p} &\multirow{3}{*}{char} & $movl \quad 8(\% eax), \% eax$\\
		&&$movl \quad (\% eax), \% al$\\
		&&$movw \quad \%al, (\%edx)$\\
	\end{tabular}
	\end{center}
  \end{table}

\section{P27}
\begin{table}[h!]
	\begin{center}
	  \begin{tabular}{c|c|c|c} % <-- Alignments: 1st column left, 2nd middle and 3rd right, with vertical lines in between
		s & c & i & d \\
		\hline
		0 & 2 & 4 & 8\\
	  \end{tabular}
	\end{center}
\end{table}
S1: 总共12字节,按4字节边界对齐。

\begin{table}[h!]
	\begin{center}
	  \begin{tabular}{c|c|c|c} % <-- Alignments: 1st column left, 2nd middle and 3rd right, with vertical lines in between
		i & s & c & d \\
		\hline
		0 & 4 & 6 & 7\\
	  \end{tabular}
	\end{center}
\end{table}
S2: 总共8字节,按4字节边界对齐。

\begin{table}[h!]
	\begin{center}
	  \begin{tabular}{c|c|c|c} % <-- Alignments: 1st column left, 2nd middle and 3rd right, with vertical lines in between
		c & s & i & d \\
		\hline
		0 & 2 & 4 & 8\\
	  \end{tabular}
	\end{center}
\end{table}
S3: 总共12字节,按4字节边界对齐。

\begin{table}[h!]
	\begin{center}
	  \begin{tabular}{c|c} % <-- Alignments: 1st column left, 2nd middle and 3rd right, with vertical lines in between
		s & c \\
		\hline
		0 & 6 \\
	  \end{tabular}
	\end{center}
\end{table}
S4: 总共8字节,按2字节边界对齐。

\begin{table}[h!]
	\begin{center}
	  \begin{tabular}{c|c|c|c|c} % <-- Alignments: 1st column left, 2nd middle and 3rd right, with vertical lines in between
		c & s & i & d & e \\
		\hline
		0 & 4 & 8 & 12 & 16\\
	  \end{tabular}
	\end{center}
\end{table}
S5: 总共24字节,按4字节边界对齐。

\begin{table}[h!]
	\begin{center}
	  \begin{tabular}{c|c|c} % <-- Alignments: 1st column left, 2nd middle and 3rd right, with vertical lines in between
		c & s & d \\
		\hline
		0 & 36 & 40\\
	  \end{tabular}
	\end{center}
\end{table}
S3: 总共44字节,按4字节边界对齐。

\section{P30}
\begin{lstlisting}
void abc(int c, long *a, int *b);
void abc(unsigned c, long *a, int *b);
void abc(long c, long *a, int *b);
void abc(unsigned long c, long *a, int *b);
\end{lstlisting}

\section{P32}
\begin{lstlisting}
typedef struct {
	int idx;
	unsigned a[6];
} line_struct;
\end{lstlisting}

由第11、12行指令可知,x数组所占空间为$0xc8-4=196B$

$\therefore LEN=\frac{196}{28}=7$
\end{document}
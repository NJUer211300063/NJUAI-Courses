% 导言区
\documentclass{article}
\usepackage{ctex}  % 添加中文包
\usepackage{multirow} % 表格单元占据多个行或者列
\usepackage{listings} % 插入代码
\usepackage{xcolor}
\lstset{    % 设置代码样式
	numbers=left, 
	numberstyle= \tiny, 
	keywordstyle= \color{ blue!70},
	commentstyle= \color{red!50!green!50!blue!50}, 
	frame=shadowbox, % 阴影效果
	rulesepcolor= \color{ red!20!green!20!blue!20} ,
	escapeinside=``, % 英文分号中可写入中文
	xleftmargin=2em,xrightmargin=2em, aboveskip=1em,
	framexleftmargin=2em
} 
\usepackage{amssymb} % 和数学公式有关的包
\title{ICS homework5} 
\author{张运吉 211300063} 
\date{7/12/2022}

% 正文区 
\begin{document} 
\maketitle % 必须有这个才能显示标题等信息

\section{P3}
\subsection*{(1)}
512MB/(64M$\times$8bit)=512MB/(64M$\times$1B)=8
所以需要8个DRAM芯片。
\subsection*{(2)}
2GB/512MB=4,需要4个内存条。
\subsection*{(3)}
按字节编址,$4GB=2^{32}B$,所以主存地址一共有32位。

设主存地址为$a_{31}...a_{0}$,因为每个内存条是512MB,所以需要29位来指出内存条内的地址,
即$a_{28}...a_{0}$,因为是64M$\times$8位的芯片,所以行列地址分别需要13位,芯片内行地址
位$a_{28}...a_{16}$,芯片内列地址位$a_{15}...a_{3}$,$a_{2}a_{1}a_{0}$用于选择芯片。
\section{P4}
\subsection*{(1)}
因为地址线有16位,所以主存地址空间为0000H~FFFFH,因为0000H~7FFFH分配给ROM区,
所以RAM区的地址范围是8000H~FFFFH,需要RAM芯片$32KB/(16K\times4bit)=4$个,地址线中最高位
即$A_{15}$用来区分ROM区和RAM区。
\subsection*{(2)}
先计算新的RAM区的大小,ROM区大小为$2^{15}B=32KB$,主存空间总大小为$16MB=512\times 32KB$,所以
RAM区大小为$511\times 32KB$,需使用的芯片数量$511\times32KB/(16K\times4bit)=2044$.
\section{P5}
数据块处理时间:$20000/500MHz=0.04ms$

磁盘旋转一圈的时间:$1/(7200/60)=0.00833s=8.33ms$,所以平均旋转等待时间为$8.33/2=4.17ms$

一个数据块的平局读取或写回时间:$4KB/(40MB/s)=0.1024ms$

所以数据块平局读取或写回时间:$2+10+4.17+0.1024=16.27ms$

因为数据块是随机存放的,所以“读出-处理-写回”一次的时间:$16.27\times 2+0.04=32.58ms$

每秒可以完成这样的操作次数:$1000/32.58\approx 30$
\section{P8}
\subsection*{(1)}
因为主存地址空间大小是1GB,所以主存地址位数一共有30位,记为$A_{29}...A_{0}$,
cache行数:$64KB/128B=512$,采取直接映射,所以cache行号占9位,每个主存块128B,
块内地址7位,主存地址标记位有$30-9-7=14$位。

综上,主存地址的高14位是标记位,中间9位是行索引,低7位是块内地址。
\subsection*{(2)}
因为采用直接映射的方式,所以不考虑替换算法,也就没有用于替换的控制位,采用直写方式,
cache中没有dirty位,所以每个cache行包含1个有效位,14个标记位,128B数据,所以,cache
总容量为$512\times (1+14+128\times 8)bit=519.5Kbit\approx 65KB$
\section{P12}
\subsection*{(1)}
数组x和y都是按存放顺序访问,所以空间局部性好,由于每个数组只访问一次,所以没有时间局部性。

无法推断命中率的高低,因为没有给出cache的容量、块大小、映射方式等信息。
\subsection*{(2)}
cache一共有两行,主存块为16字节,float类型四个字节,所以一个主存块可以包含4个数组元素,
数组x存放在40H开始的连续字节中,x[0]-x[3]在内存中属于第四块,x[4]-x[7]在第5块,
y[0]-y[3]属于第六块,y[4]~y[7]属于第七块,所以x[0]-x[3]与y[0]-y[3]总是映射到cache第0行,
x[4]-x[7]与y[4]~y[7]总是映射到cache第1行,每个x[i]和y[i]总是命中到同一行,互相淘汰对方,命中率为0.
\subsection*{(3)}
cache一共有四行,分为两个组,x[0]-x[1]属于第8块,依此类推,y[6]-y[7]存放在第15块,
因为每个组有两行,所以x[i]和y[i]可以存放在同一组的不同行中,对于每一块中的两个数组元素,
在访问第一个元素总是未命中,访问第二个总是命中,所以命中率是$50\%$.
\subsection*{(4)}
若是将x定义为float[12],则使得y从主存第七块开始存放,那么x[i]和y[i]存放cache在不同行中,不会相互淘汰,
对于每一块中的四个数组元素,第一个总是未命中,后面3个总是命中,所以命中率为$75\%$.
\section{P13}
当cache数据区大小为32B时:
\begin{table}[h]
	\begin{center} % 表格居中
	  \begin{tabular}{c|c|c|c|c|c|c|c|c} % <-- Alignments: 1st column left, 2nd middle and 3rd right, with vertical lines in between
		\hline
		\textbf{} & \textbf{col=0} & \textbf{col=1} & \textbf{col=2} & \textbf{col=3} &\textbf{col=0} & \textbf{col=1} & \textbf{col=2} & \textbf{col=3}\\
		\hline
		row=0 & 0/miss & 0/miss & 0/hit & 0/miss & 0/miss & 0/miss & 0/miss & 0/miss \\
		\hline
		row=1 & 1/miss & 1/hit & 1/miss & 1/hit & 1/miss & 1/miss & 1/miss & 1/miss \\
		\hline
		row=2 & 0/miss & 0/miss & 0/hit & 0/miss & 0/miss & 0/miss & 0/miss & 0/miss \\
		\hline
		row=3 & 1/miss & 1/hit & 1/miss & 1/hit & 1/miss & 1/miss & 1/miss & 1/miss \\
		\hline
	\end{tabular}
	\end{center}
\end{table}

当cache数据区大小为128B时:
\begin{table}[h]
	\begin{center} % 表格居中
	  \begin{tabular}{c|c|c|c|c|c|c|c|c} % <-- Alignments: 1st column left, 2nd middle and 3rd right, with vertical lines in between
		\hline
		\textbf{} & \textbf{col=0} & \textbf{col=1} & \textbf{col=2} & \textbf{col=3} &\textbf{col=0} & \textbf{col=1} & \textbf{col=2} & \textbf{col=3}\\
		\hline
		row=0 & 4/miss & 4/hit & 4/hit & 4/hit & 0/miss & 0/hit & 0/hit & 0/hit \\
		\hline
		row=1 & 5/miss & 5/hit & 5/hit & 5/hit & 1/miss & 1/hit & 1/hit & 1/hit \\
		\hline
		row=2 & 6/miss & 6/hit & 6/hit & 6/hit & 2/miss & 2/hit & 2/hit & 2/hit \\
		\hline
		row=3 & 7/miss & 7/hit & 7/hit & 7/hit & 3/miss & 3/hit & 3/hit & 3/hit \\
		\hline
	\end{tabular}
	\end{center}
\end{table}
\section{P19}
\subsection*{(1)}
直接映射,s=64:访存顺序为:a[0],a[64],a[0],a[64] ...

a[0]和a[64]都是对应主存块的第一个元素,a[0]和a[64]相差256B,正好是8个主存块,因此这两个元素所在内存块
会映射到cache同一行,每次都会发生冲突,所以缺失率为$100\%$.
\subsection*{(2)}
直接映射,s=63:访存顺序为:a[0],a[63],a[126],a[0],a[63],a[126]...

a[63]所在主存块的第一个元素是a[56],a[126]所在主存块的第一个元素是a[120],a[56]与a[120]
正好相差256B,即8个主存块,所以会映射到cache同一行,每次都会发生冲突,而a[0]则不会发生冲突,
所以缺失率是$67\%$.
\subsection*{(3)}
2路组相联,s=64:访存顺序为:a[0],a[64],a[0],a[64] ...

a[0]和a[64]都是对应主存块的第一个元素,a[0]和a[64]相差256B,正好是8个主存块,因此这两个元素所在内存块
会映射到cache同一组,但由于是两路组相联,所以可以放在同一组的不同行中,只有第一次访问时会缺失,后续访问不会发生冲突,冲突率几乎是$0$。
\subsection*{(4)}
2路组相联,s=63:访存顺序为:a[0],a[63],a[126],a[0],a[63],a[126]...

a[63]所在主存块的第一个元素是a[56],a[126]所在主存块的第一个元素是a[120],a[56]与a[120]
正好相差256B,即8个主存块,所以会映射到cache同一组,但由于是两路组相联,所以可以放在同一组的不同行中,只有第一次访问时会缺失,后续访问不会发生冲突,
a[0]则存放在别的组中,不会发生冲突,总的来说,只有第一次访问时会发生冲突,后续访问不会发生冲突,冲突率几乎是$0$。
\section{P20}
由于$2^{36}B/16KB=2^{22}$,物理页号的位数至少为22,每个页表项包括有效位、保护位、修改位、使用为、物理页号等,
所以一个页表项至少为:$22+4=26位$,虚拟页数为:$2^{40}B/16KB=2^{26}$,为了简化对页表的访问,页表项的大小应是2
的幂次,这里我们取32,所以每个进程的页表大小为:$2^{26}\times 32bit = 256MB$.

问题:构建出的页表太大,导致无法一次性加入到内存中。
\section{P21}
\subsection*{(1)}
低7位是页内偏移量,高9位是虚页号,虚页号中高7位是TLB标记,低2位是TLB组索引。
\subsection*{(2)}
低7位是页内偏移量,高5位是物理页号。
\subsection*{(3)}
低2位是块内地址,中间4位是cache行索引,高6位是标记。
\subsection*{(4)}
虚拟地址:067AH=0000 0110 0111 1010B,所以虚页号为:000001100B,低2为是00B,
所以映射到TLB中第0组,高7位标记是0000011B=03H,TLB中对应的表项有效位是0,所以
TLB缺失,需要到内存的页表中查找,查找b图中的00C处表项,有效位是1,说明对应的页在内存中,
取出物理页号19H,和虚拟地址中的页内偏移组合形成物理地址:110011 1110 10B,去除物理地址中的
cache行索引字段1110,直接找到cache中第E行,其有效位是1,且标记33H=110011B,cache命中。
根据物理地址的低2位10B,取出字节2中的内容2D4AH。
\section{P23}
\subsection*{(1)}
$movl \quad \$0,\%ecx$

$.loop$

$cmpl \quad \%ebx,\%ecx$

$jge \quad .done$

$addl \quad (\%edx,\%ecx,4), \%eax$

$incl \quad \%ecx$

$jmp \quad .loop$

$.done$
\subsection*{(2)}
执行到程序P时,处于保护模式并且采用分页虚拟管理方式,所以$PE=1, PG=1$.
\subsection*{(3)}
IA-32中long时32位带符号整数,所以指令I为$addl \quad (\%edx,\%ecx,4), \%eax$,
寻址方式是"基址+比例变址+偏移量"。
\subsection*{(4)}
线性地址:$0x0+0x8048c08=0x8048c08$.

指令I的地址低12位是页内偏移量:1100 0000 1000,高20位是虚页号,虚页号是:0000 1000 0000 0100 1000.

虚页号中低10位是页表索引:00 0100 1000,高10位是页目录索引:0000 1000 00.

指令I在内存地址为:$0x1020000+0xC08=0x1020C08$起的4个字节。

P=1,R/W=0,U/S=1,A=1,D=0.
\subsection*{(5)}
取指令不会发生缺页异常,因为指令I不在一个页的起始处,在执行指令I前面的指令发生缺页时,
会将指令I一起调入内存;但在取操作数a[0]的时候可能会发生缺页异常,因为a[0]所在地址对应的
页可能还没有加载到内存,若是这样,页故障线性地址是0x8049000,该地址保存在控制寄存器CR2中。
\subsection*{(6)}
同(5),取指令时不会发生TLB缺失,但取操作数可能会发生TLB缺失。
20为虚拟页号中高18位位TLB标记,低2位为TLB索引。

根据TLB标记和TLB索引,在TLB中能找到一个标记一样且有效位为1的的页表项,TLB命中,取出
页框号028B0H,得到主存地址0x28b0c08.
\subsection*{(7)}
指令cache一共有$8KB/32B=256$行,组数为128。指令I的线性地址为0x8048c08,其中第12为时页内偏移量,
,组索引为1100000,块内地址为01000,因为指令I不在某个主存块的起始位置,因此在第一次执行指令I时,
不会发生cache缺失。

映射到1100000组中。
\subsection*{(8)}
N=2000,数组a的大小为8000B,链接后a的首地址为0x8049300,对应段基址为0,所以a的
线性地址为0x8049300,这不是一个页面的起始地址,$8000/4K=\approx 1.9$,最后占三个页面。
虚页号分别是0000 1000 0000 0100 1001和0000 1000 0000 0100 1010和0000 1000 0000 0100 1011.

因为$4\times 1200=4800>4096$,所以a[1200]在第二个页面。
\end{document}
% 导言区
\documentclass{article}
\usepackage{ctex}  % 添加中文包
\usepackage{multirow} % 表格单元占据多个行或者列
\usepackage{listings} % 插入代码
\usepackage{xcolor}
\lstset{    % 设置代码样式
	numbers=left, 
	numberstyle= \tiny, 
	keywordstyle= \color{ blue!70},
	commentstyle= \color{red!50!green!50!blue!50}, 
	frame=shadowbox, % 阴影效果
	rulesepcolor= \color{ red!20!green!20!blue!20} ,
	escapeinside=``, % 英文分号中可写入中文
	xleftmargin=2em,xrightmargin=2em, aboveskip=1em,
	framexleftmargin=2em
} 
\usepackage{amssymb} % 和数学公式有关的包
\title{ICS homework3} 
\author{张运吉 211300063} 
\date{16/11/2022}

% 正文区 
\begin{document} 
\maketitle % 必须有这个才能显示标题等信息

\section*{P3}

\begin{table}[h]
	\begin{center} % 表格居中
	  \begin{tabular}{c|c|c|c|c} % <-- Alignments: 1st column left, 2nd middle and 3rd right, with vertical lines in between
		\hline
		\textbf{符号} & \textbf{是否在test.o的符号表中} & \textbf{定义模块} & \textbf{符号类型} & \textbf{节}\\
		\hline
		a & 是 & main.o & 外部 & .data \\
		\hline
		val & 是 & test.o & 全局 & .data \\
		\hline
		sum & 是 & test.o & 全局 & .text \\
		\hline
		i & 不是 & &  &  \\
		\hline
		
	  \end{tabular}
	\end{center}
\end{table}

\section*{P4}

\begin{table}[h]
	\begin{center} % 表格居中
	  \begin{tabular}{c|c|c|c|c} % <-- Alignments: 1st column left, 2nd middle and 3rd right, with vertical lines in between
		\hline
		\textbf{符号} & \textbf{是否在swap.o的符号表中} & \textbf{定义模块} & \textbf{符号类型} & \textbf{节}\\
		\hline
		buf & 是 & main.o & 外部 & .data \\
		\hline
		bufp0 & 是 & swap.o & 全局 & .data \\
		\hline
		bufp1 & 是 & swap.o & 本地 & .bss \\
		\hline
		incr & 是 & swap.o & 本地 & .text \\
		\hline
		count & 是 & swap.o & 本地 & .data \\
		\hline
		swap & 是 & swap.o & 全局 & .text \\
		\hline
		temp & 不是 & &  &  \\
	  \end{tabular}
	\end{center}
\end{table}

\section*{P5}
\subsection*{(1)}
main.c中的强符号有x、z、main,弱符号有y、proc1;

proc1.c中的强符号有proc1,弱符号有x。

\subsection*{(2)}
x被分配在main.o的.data节中,占4个字节,其后是z,占2个字节,z后的2字节空余。
所以,执行第7行前&x和&z中存放的内容如下:
\begin{table}[h]
	\begin{center} % 表格居中
	  \begin{tabular}{c|c|c|c|c} % <-- Alignments: 1st column left, 2nd middle and 3rd right, with vertical lines in between
		\textbf{} & \textbf{0} & \textbf{1} & \textbf{2} & \textbf{3}\\
		\hline
		\&z & 02 & 00 & ... & ... \\
		\hline
		\&x & 01 & 01 & 00 & 00 \\
		
	  \end{tabular}
	\end{center}
\end{table}

根据符号解析的规则,在执行proc1时,x绑定的是main.o中定义的x,所以$x=-1.5$实际上做的是把-1.5的机器数放到&x开始的8个字节,执行第7行前&x和&z中存放的内容如下:
\begin{table}[h]
	\begin{center} % 表格居中
	  \begin{tabular}{c|c|c|c|c} % <-- Alignments: 1st column left, 2nd middle and 3rd right, with vertical lines in between
		\textbf{} & \textbf{0} & \textbf{1} & \textbf{2} & \textbf{3}\\
		\hline
		\&z & 00 & 00 & F8 & BF \\
		\hline
		\&x & 00 & 00 & 00 & 00 \\
		
	  \end{tabular}
	\end{center}
\end{table}

所以最后打印的结果是"x=0,z=0".

若第3行改为"short y=1, z=2;"执行第7行前.data中存放的内容如下:
\begin{table}[h]
	\begin{center} % 表格居中
	  \begin{tabular}{c|c|c|c|c} % <-- Alignments: 1st column left, 2nd middle and 3rd right, with vertical lines in between
		\textbf{} & \textbf{0} & \textbf{1} & \textbf{2} & \textbf{3}\\
		\hline
		\&y & 01 & 00 & 02 & 00 \\
		\hline
		\&x & 01 & 01 & 00 & 00 \\
		
	  \end{tabular}
	\end{center}
\end{table}

执行第7行后.data中存放的内容如下:
\newpage
\begin{table}[h]
	\begin{center} % 表格居中
	  \begin{tabular}{c|c|c|c|c} % <-- Alignments: 1st column left, 2nd middle and 3rd right, with vertical lines in between
		\textbf{} & \textbf{0} & \textbf{1} & \textbf{2} & \textbf{3}\\
		\hline
		\&y & 00 & 00 & F8 & BF \\
		\hline
		\&x & 00 & 00 & 00 & 00 \\
		
	  \end{tabular}
	\end{center}
\end{table}

所以最好打印的结果是"x=0,z=-16392".

\subsection*{(3)}

将模块proc1.c中第1行修改为"static double x;"就会使x被定义为本地符号,从而在调用proc1的时候x被重定位到proc1.c中定义的x,就不会修改main.c中定义的x的值,从而得到期望的结果。


\section*{P6(3)}
1.REF(m1.main)$\longrightarrow$在m1中没有对main的引用

2.REF(m2.main)$\longrightarrow$DEF(m1.main)

3.REF(m1.p1)$\longrightarrow$DEF(m2.p1)

4.REF(m1.x)$\longrightarrow$这里的x是局部变量

5.REF(m2.x)$\longrightarrow$DEF(m2.x)


\section*{P7}

m1.c中定义的main是强符号,p1是弱符号,m2.c中定义的main是弱符号,p1是强符号。
执行main函数时,调用的p1是在m2.c中定义的p1,而m2.c中的p1函数中打印的main是m1.c中的main,
即输出main函数起始地址的两个字节内容,因为main函数的前两条机器指令是"55"和"89 e5",
所以最后会打印出字符串"0x5589\n".


\section*{P8}

可读写数据段由所有可重定位目标文件中的.data节和.bss节合并形成,.data节中存的是初始化的全局变量,必须在可执行文件中记录它们的值,
而.bss节存的是未初始化的全局变量,因此可执行文件中没有记录它们的值。

由图可知,只需将可执行文件中偏移地址0x448开始的0xe8个字节存放到对应的虚拟地址空间中,所以
0xe8是可执行文件中.data节的大小,后面的28字节是未初始化的全局变量所在的区域。

\section*{P9}
(1) gcc -static -o a p.o libx.o liby.o

(2) gcc -static -o a p.o libx.o liby.o libx.o

(3) gcc -static -o a p.o libx.o liby.o libx.o libz.o

\section*{P10}
需重定位的符号名:swap

相对.text节起始位置的偏移:0x7

所在指令行号:6

重定位类型: R\_386\_PC32

重定位前的内容:$0xfffffffc=-4$

重定位后的内容:$07 00 00 00$

计算过程:main函数最后一条指令地址:$0x8048386+0x12=0x8048398$,swap代码紧跟
其后,按4字节边界对齐,所以swap的首地址为$0x8048398$,所以偏移量$=0x8048398-(0x8048386+0x7)-(-4)=7$
\end{document}
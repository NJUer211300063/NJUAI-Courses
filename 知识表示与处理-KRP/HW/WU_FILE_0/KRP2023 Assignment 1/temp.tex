\documentclass[a4paper]{article}
% \usepackage[margin=1.25in]{geometry}
\usepackage[inner=2.0cm,outer=2.0cm,top=2.5cm,bottom=2.5cm]{geometry}
\usepackage{ctex}
\usepackage{color}
\usepackage{graphicx}
\usepackage{amssymb}
\usepackage{amsmath}
\usepackage{amsthm}
\usepackage{bm}
\usepackage{hyperref}
\usepackage{multirow}
\usepackage{mathtools}
\usepackage{enumerate}

\newcommand{\homework}[5]{
	\pagestyle{myheadings}
	\thispagestyle{plain}
	\newpage
	\setcounter{page}{1}
	\noindent
	\begin{center}
		\framebox{
			\vbox{\vspace{2mm}
				\hbox to 6.28in { {\bf 计算方法 \hfill #2} }
				\vspace{6mm}
				\hbox to 6.28in { {\Large \hfill #1 \hfill} }
				\vspace{6mm}
				\hbox to 6.28in { {\it Instructor: {\rm #3} \hfill Name: {\rm #4}, StudentId: {\rm #5}}}
				\vspace{2mm}}
		}
	\end{center}
	% \markboth{#4 -- #1}{#4 -- #1}
	\vspace*{4mm}
}


\newenvironment{solution}
{\color{blue} \paragraph{Solution.}}
{\newline \qed}

\begin{document}
	%==========================Put your name and id here==========================
	\homework{Homework 1}{Spring 2023}{Lijun Zhang}{张运吉}{211300063}
	

	
	\paragraph{第2题}
	
	\begin{solution}
		由泰勒公式:$e(x^n) \approx n{x^*}^{n-1}(x^*-x)$
		
		$\therefore e_{r}(x^n) \approx {\frac{e(x^n) }{(x^*)^n}} = n \frac{x^*-x}{x^*} = ne_{r}(x) = 0.02n$
	\end{solution}
	
	
	
	
	\paragraph{第5题}
	\begin{solution}
		$f(R) = \frac{4}{3} \pi R^3, e^{*}(f(R)) = 4\pi R^{*3}(R^*-R)$
		
		$\therefore e_{r}^{*}(f(R)) =  e^{*}(f(R)) / f(R^{*}) = 3(R^*-R) = 3e^{*}(R) \leq 1 \%$
		
		$\therefore e^{*}(R) \leq 1 / 300$
		\end{solution}
	
	\paragraph{第8题}
	\begin{solution}
		令$x = arctanA, y = arctanB, tan(x-y) = \frac{tanx-tany}{1+tan(x)tan(y)} = \frac{A-B}{1+AB}$
		
		$\therefore x - y = arctan(\frac{A-B}{1+AB})$, 即 $arctanA - arctanB = arctan(\frac{A-B}{1+AB})$
		
		$\int_{N}^{N+1} \frac{1}{x^2+1} = arctan(N+1) - arctan(N) = arctan(\frac{1}{1+N(N+1)})$
		\end{solution}
	

	
	\paragraph{第13题}
	\begin{solution}
		令$y=x-\sqrt{x^2-1}$
		
		$\sqrt{899} \approx 29.9833, 30 - 29.9833 \approx 0.0167$
		
		$\therefore y^* = 0.0167, |y^* - y| \leq \frac{1}{2} \times 10^{-4}$
		
		$\therefore \epsilon(f(x)) \approx \frac{1}{|y^*|} |y^* - y|\leq \frac{\frac{1}{2} \times 10^{-4}}{0.0167} \leq 0.3 \times 10^{-2} $
		
		若用等价公式$ln(x-\sqrt{x^2-1}) = -ln(x+\sqrt{x^2-1})$
		
		则$y^* = 59.9833$
		
		$\therefore \epsilon(f(x)) \approx \frac{1}{|y^*|} |y^* - y|\leq \frac{\frac{1}{2} \times 10^{-4}}{59.9833} \leq 8.34 \times 10^{-7}$
		\end{solution}
	
	\paragraph{第1题}
\begin{solution}
  注意到右端项是$y=x$在$x_0=0, x_1=1, ...,x_n=n$处的Lagrange插值多项式。
  
  $\therefore x \approx \sum \limits_{i=0}^{n} l_i(x)x_i = \sum\limits_{i=0}^{n}(\prod\limits_{k=0,k\neq i}^{n}\frac{x-k}{i-k})i$

  余项是$R_n(x) =x - \sum \limits_{i=0}^{n} l_i(x)x_i= \frac{x^{(n+1)}|_{x=\xi}}{(n+1)!}\omega_{n+1}(x)=0$
  
  $\therefore x=\sum\limits_{i=0}^{n}(\prod\limits_{k=0,k\neq i}^{n}\frac{x-k}{i-k})i$
\end{solution}

	\paragraph{第2题}
\begin{solution}
	令$x_0=1,y_0=0;x_1=-1,y_1=-3;x_2=2,y_2=4$
	
	使用Lagrange插值法,
	
	$l_0(x) = \frac{(x-x_1)(x-x_2)}{(x_0-x-1)(x_0-x_2)}=-\frac{1}{2}(x+1)(x-2)$
	
	$l_1(x) = \frac{(x-x_0)(x-x_2)}{(x_1-x_0)(x_1-x_2)}=\frac{1}{6}(x-1)(x-2)$
	
	$l_2(x) = \frac{(x-x_0)(x-x_1)}{(x_2-x_0)(x_2-x_1)}=\frac{1}{3}(x-1)(x+1)$
	
	$\therefore L_2(x)=\sum\limits_{i=0}^{2}y_il_i(x) = \frac{5}{6}x^{2}+\frac{3}{2}x-\frac{7}{3}$
	
	使用Newton插值法,
	
	一阶差商:$f[x_0,x_1]=\frac{3}{2}, f[x_1,x_2]=\frac{7}{3}$
	
	二阶差商: $f[x_0, x_1, x_2] = \frac{5}{6}$
	
	$\therefore N_2(x) = y_0 + f[x_0, x_1](x-x_0) + f[x_0, x_1, x_2](x-x_0)(x-x_1) = \frac{5}{6}x^{2}+\frac{3}{2}x-\frac{7}{3}$
\end{solution}

	\paragraph{第4题}
\begin{solution}
 \leavevmode \\
	当$x\in [x_k, x_{k+1}]$时,线性插值多项式为:
	\begin{equation}
		L_1(x) = \cos x_k \frac{x-x_{k+1}}{x_k-x_{k+1}}+\cos x_{k+1}\frac{x-x_k}{x_{k+1}-x_k} \notag
	\end{equation}
	其中$x_k = \frac{k}{60} \times \frac{\pi}{180} = \frac{k\pi}{10800}$.
 
误差估计
	\begin{equation}
		\begin{split}
			|\cos x - L'_1(x)| &= |\cos x - L_1(x) + L_1(x) - L'_1(x)| \\
			&\leq |\cos x - L_1(x)| + |L_1(x)-L'_1(x)|
		\end{split}
		\notag
	\end{equation}
	将误差估计分为两部分分别计算
	\begin{equation}
		\begin{split}
			|\cos x - L_1(x)| &= \left| \frac{1}{2}(-\cos \xi)(x-x_k)(x-x_{k+1}) \right| \\
			&\leq \frac{1}{2} |(x-x_k)(x-x_{k+1})| \\
			&\leq \frac{1}{2}\times \left( \frac{1}{2} \times \frac{\pi}{10800} \right) ^2 \\
			&\approx 1.06 \times 10^{-8}
		\end{split}
		\notag
	\end{equation}
	\begin{equation}
		\begin{split}
			|L_1(x)-L'_1(x)| &= |e(f^*(x_k))|\frac{x_{k+1}-x}{x_{k+1}-x_k} + |e(f^*(x_{k+1}))|\frac{x-x_k}{x_{k+1}-x_k}\\
			&\leq \max \{|e(f^*(x_k))|, |e(f^*(x_{k+1}))|\} \left( \frac{x_{k+1}-x}{x_{k+1}-x_k} + \frac{x-x_k}{x_{k+1}-x_k} \right) \\
			&= \max \{|e(f^*(x_k))|, |e(f^*(x_{k+1}))|\}
		\end{split}
		\notag
	\end{equation}
	由有效数字的定义可得
	\begin{equation}
		|e(f^*(x_k))| \leq \frac{1}{2}\times 10^{m_k-4} \notag
	\end{equation}
	所以有
	\begin{equation}
		\max \{|e(f^*(x_k))|, |e(f^*(x_{k+1}))|\} \leq \max \left\{ \frac{1}{2}\times 10^{m_k-4}, \frac{1}{2}\times 10^{m_{k+1}-4} \right\} = \frac{1}{2}\times 10^{\max \{m_k, m_{k+1}\}-4} \notag
	\end{equation}
	综上所述
	\begin{equation}
		|\cos x-L'_1(x)| \leq 1.06\times 10^{-8} + \frac{1}{2}\times 10^{\max \{m_k, m_{k+1}\}-4} \notag
	\end{equation}
	在区间 $[0, \frac{\pi}{2}]$上可得
	\begin{equation}
		|\cos x-L'_1(x)|\leq 1.06 \times 10^{-8} + \frac{1}{2}\times 10^{-5} = 0.50106 \times 10^{-5} \notag
	\end{equation}
\end{solution}

	\paragraph{第6题}
\begin{solution}
		i).注意到等式左边是$y=x^k$在$(x_j, x_j^k), j=0,1,...,n$处的Lanrange插值多项式
	
	$\therefore x^k\approx \sum\limits_{j=0}^{n}x_j^kl_j(x)$
	
	$R_n(x) = x^k - \sum\limits_{j=0}^{n}x_j^kl_j(x) = \frac{(x^k)^{(n+1)}|_{x=\xi}}{(n+1)!}\omega_{n+1}(x) = 0$
	
	$\therefore \sum\limits_{j=0}^{n}x_j^kl_j(x)\equiv x^k$
	
	ii).\begin{equation*}
		\begin{aligned}
			\sum\limits_{j=0}^{n} (x_j-x)^kl_j(x) &= \sum\limits_{j=0}^{n}\left[ l_j(x)\sum\limits_{i=0}^{k}\tbinom{k}{i}x_j^{i}(-x)^{k-i} \right] \\
			&= \sum\limits_{j=0}^{n} \sum\limits_{i=0}^{k} \left[\tbinom{k}{i}x_j^{i}(-x)^{k-i}l_j(x)\right] \\
			&= \sum\limits_{i=0}^{k} \sum\limits_{j=0}^{n} \left[\tbinom{k}{i}x_j^{i}(-x)^{k-i}l_j(x)\right] \\ 
			&= \sum\limits_{i=0}^{k} \left[\tbinom{k}{i}(-x)^{k-i}\sum\limits_{j=0}^{n}x_j^{i}l_j(x)\right]\\ 
			&= \sum\limits_{i=0}^{k} \tbinom{k}{i}(-x)^{k-i}x^{i} \\
			&= (x-x)^{k}\equiv 0\\
		\end{aligned}
	\end{equation*}
\end{solution}

	\paragraph{第8题}
\begin{solution}
	根据截断误差公式:$R_{n}(x)=f(x)-L_{n}(x)=\frac{f^{(n+1)}(\xi)}{(n+1) !} \omega_{n+1}(x)$

$\begin{array}{l}R_{2}(x)=\frac{1}{3 !} f^{\prime \prime \prime}(\xi)\left(x-x_{i-1}\right)\left(x-x_{i}\right)\left(x-x_{i+1}\right), \quad \xi \in\left(x_{i-1},\right. \\ \left.x_{i+1}\right)\end{array}$

	其中:$x_{i-1}=x_i-h, x_{i+1}=x_i+h$

	\begin{aligned}\left|R_{2}(x)\right| & = \frac{1}{6}e^x\left|(x-x_{i-1})(x-x_i)(x-x_{i+1})\right|& \leqslant \frac{1}{6} \mathrm{e}^{4} \max _{x_{i-1} \leqslant x \leqslant x_{i+1}}\left|\left(x-x_{i-1}\right)\left(x-x_{i}\right)\left(x-x_{i+1}\right)\right| \\ & \leqslant \frac{1}{6} \mathrm{e}^{4} \frac{2}{3} \frac{1}{\sqrt{3}} h^{3}=\frac{\mathrm{e}^{4}}{9 \sqrt{3}} h^{3}\end{aligned}

 其中第二个不等式根据求导得到。

 令$\frac{\mathrm{e}^{4}}{9 \sqrt{3}} h^{3}\leq10^{-6}$,得到$h\leq0.00658$
\end{solution}

	\paragraph{第17题}
\begin{solution}
\leavevmode \\
由Hermite插值函数的条件可知
\begin{equation}
    H_3(x_k) = f_k, H_3(x_{k+1})=f_{k+1},  H'_3(x_k) = f'_k, H'_3(x_{k+1}) = f'_{k+1} \notag
\end{equation}
由此可知: $R_3(x)$ 有二重零点$x_K,X_{k+1}$,则
\begin{equation}
    R_3(x) = f(x) - H_3(x) = g(x)(x-x_k)^2(x-x_{k+1})^2 \notag
\end{equation}
令 $h(t) = f(t) - H_3(t) -g(t)(t-x_k)^2(t-x_{k+1})^2。

则 $h(x_k)=h(x_{k+1})=0,h'(x_k)=h'(x_{k+1})=0,h(x)=0$

在$[x_k,x]$和$[x,x_{k+1}]$上对$h(x)$使用Rolle中值定理可得 $\exists \xi_1 \in [x_k,x], \quad \exists \xi_2 \in [x, x_{k+1}]$使得 $h'(\xi_1) = h'(\xi_2) = 0$

在$[x_k,\xi_1],[\xi_1,\xi_2]$和$[\xi_2,x_{k+1}]$对$h'(x)$使用Rolle定理可得
$\exists \xi_{11} \in [x_k, \xi_1], \xi_{22} \in [\xi_1, \xi_2], \xi_{33} \in [\xi_2, x_{k+1}]$使得 $h''(\xi_{11}) = h''(\xi_{22}) = h''(\xi_{33}=0)$。

同理再用两次Rolle定理,可得$h^{(4)}(\xi) = f^{(4)}(t)-k(x)\times 4!$,可解得
\begin{equation}
    k(x) = \frac{1}{4!}f^{(4)}(\xi) \notag
\end{equation}
所以有
\begin{equation}
    R_3(x)=\frac{1}{4!}f^{(4)}(\xi)(x-x_k)^2(x-x_{k+1})^2 \notag
\end{equation}
令 $x_k = a+kh, h = \frac{b-a}{n}$,在 $[x_k, x_{k+1}]$上有
\begin{equation}
    \begin{split}
        |f(x)-H_3(x)| &= \frac{1}{4!}|f^{(4)}(x)|(x-x_k)^2(x-x_{k+1})^2 \\
        &\leq \frac{1}{4!}\max |f^{(4)}(x) | \max (x-x_k)^2(x-x_{k+1})^2
    \end{split}
    \notag
\end{equation}
由于
\begin{equation}
    \max (x-x_k)^2(x-x_{k+1})^2 = \max (s^2(s-1)^2h^4) = \frac{1}{16}h^4
\end{equation}
误差估计为
\begin{equation}
    |f(x)-I_h(x)| \leq \frac{1}{384}h^4\max_{a\leq x \leq b}|f^{(4)}(x)|
\end{equation}
\end{solution}
	
	\paragraph{第19题}
\begin{solution}
$x_0=0,y_0=0,m_0=0;x_1=1,y_1=1,m_1=1$

满足上面条件的Hermite插值多项式为:

$\begin{array}{l}H_{3}(x)=\sum_{j=0}^{1}\left[y_j \alpha_{j}(x)+m_j \beta_{j}(x)\right] \\ =\left[1-2 \frac{x-1}{1-0}\right]\left[\frac{x-0}{1-0}\right]^{2}+(x-1)\left[\frac{x-0}{1-0}\right]^{2}=2 x^{2}-x^{3}\end{array}$

令$P(x)=H_{3}(x)+a x^{2}(x-1)^{2}$,

由$P(2)=1$,解得$a=\frac{1}{4}$

$\therefore P(x)=2 x^{2}-x^{3}+\frac{1}{4} x^{2}(x-1)^{2}=\frac{1}{4} x^{2}(x-3)^{2}$
\end{solution}

\end{document}

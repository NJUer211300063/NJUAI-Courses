\documentclass[12pt, a4paper, oneside]{ctexart}
\usepackage{amsmath, amsthm, amssymb, graphicx, fontspec, listings, algorithm, algpseudocode, algorithmicx, enumerate}
\usepackage[bookmarks=true, colorlinks, citecolor=blue, linkcolor=black]{hyperref}
\usepackage{graphicx}
\usepackage{hyperref}
\setmainfont{Times New Roman}

\CTEXsetup[format={\Large\bfseries}]{section}
\lstset{
 columns=fixed,       
 numbers=left,                                        % 在左侧显示行号
 numberstyle=\tiny\color{gray},                       % 设定行号格式
 frame=none,                                          % 不显示背景边框
 backgroundcolor=\color[RGB]{245,245,244},            % 设定背景颜色
 keywordstyle=\color[RGB]{40,40,255},                 % 设定关键字颜色
 numberstyle=\footnotesize\color{black},           
 commentstyle=\it\color[RGB]{0,96,96},                % 设置代码注释的格式
 stringstyle=\rmfamily\slshape\color[RGB]{128,0,0},   % 设置字符串格式
 showstringspaces=false,                              % 不显示字符串中的空格
 language=c++,                                        % 设置语言
}

\title{第一章作业} 
\author{211300059 郭霄飏}
\date{\today}

\begin{document}
\maketitle
\section*{第一章}
\section*{2.}
\noindent 由泰勒展开可得
\begin{equation}
    e(x^n) \approx nx^{n-1}(x-x^*) \notag
\end{equation}
根据相对误差的定义可得
\begin{equation}
    \begin{split}
        e_r(x^n) &\approx \frac{nx^{n-1}(x-x^*)}{x^n} \\
        &= \frac{n(x-x^*)}{x^*} \\
        &= ne_r(x) \\
        &= 0.02n \notag
    \end{split}
\end{equation}


\section*{5.}
\noindent 令 $V = \frac{3}{4} \pi R^3$,可得
\begin{equation}
        e^*(V) \approx 4\pi R^2(R^*-R) \notag
\end{equation}
由此可得
\begin{equation}
    e_r^*(V)=\frac{e^*(V)}{\frac{4}{3}\pi R^3}=\frac{3(R^*-R)}{R^*} = 3e_r^*(R) \notag
\end{equation}
所以
\begin{equation}
    \begin{split}
        3e_r^*(R) &\leq 1\% \\
        e_r^*(R) &\leq \frac{1}{300}
    \end{split}
    \notag
\end{equation}

\section*{8.}
\noindent
\begin{equation}
    \begin{split}
        \int_{N}^{N+1} \frac{1}{x^2+1} = \arctan (N+1) - \arctan(N) = \arctan(\frac{1}{1+N(N+1)}) \notag
    \end{split}
\end{equation}

\section*{13.}
\noindent 
令 $y = x - \sqrt{x^2-1}$,则有
\begin{equation}
    \begin{split}
        \xi (f(x^*)) \approx \frac{1}{|y^*|}|y^*-y| \notag
    \end{split}
\end{equation}
由题可知,开根号使用六位有效数字,所以 $\sqrt{30^2 -1} \approx 29.9833 \quad |y^*| = 0.0167$.
又由教材公式2.2, $|y^*-y| \leq \frac{1}{2} \times 10^{-4}$,所以
\begin{equation}
    \xi(f(x^*)) \approx \frac{|y^*-y|}{|y^*|} \leq \cfrac{\frac{1}{2} \times 10^{-4}}{0.0167} \approx 0.003 \notag
\end{equation}
若用公式 $-\ln (x+\sqrt{x^2-1})$,同理可得 $|y*|=59.9833$,所以
\begin{equation}
    \xi (f(x^*)) \approx \frac{|y^*-y|}{|y^*|}\leq \frac{\frac{1}{2}\times 10^{-4}}{59.8333} \approx 8.34 \times 10^{-7} \notag
\end{equation}


\section*{第二章}
\section*{1.}
\noindent
将右侧与函数 $f(x)=x$的Lagrange插值多项式联系
\begin{equation}
    x \approx \sum_{i=0}^{n}x_i l_i(x) = \sum_{i=0}^{n} \left( \prod_{k=0, k \neq i} \frac{x-k}{i-k} \right)i \notag
\end{equation}
由插值余项
\begin{equation}
    R_n(x)=f(x)-L_n(x)=\frac{x^{(n+1)}(\xi)}{(n+1)!}\omega_{n+1}(x) = 0 \notag
\end{equation}
由此可得
\begin{equation}
    x = \sum_{i=0}^{n}x_i l_i(x) = \sum_{i=0}^{n} \left( \prod_{k=0, k \neq i} \frac{x-k}{i-k} \right)i \notag
\end{equation}

\section*{2.}
\noindent
拉格朗日插值
\begin{equation}
    \begin{split}
        L_2(x) &= \sum_{k=0}^{n}f_k(x_k)l_0(x) \\
        &= 0 + (-3) \times \frac{(x-1)(x-2)}{-2\times (-3)} + 4\times \frac{(x-1)(x+1)}{1\times 3} \\
        &= \frac{5}{6}x^2 + \frac{3}{2}x - \frac{7}{3}
    \end{split}
    \notag
\end{equation}




\section*{4.}
\noindent
线性插值多项式为 
\begin{equation}
    L_1(x) = \cos x_k \frac{x-x_{k+1}}{x_k-x_{k+1}}+\cos x_{k+1}\frac{x-x_k}{x_{k+1}-x_k} \notag
\end{equation}
令 $L'_1(x)$为近似值线性插值多项式,$x_k = \frac{k}{60} \times \frac{\pi}{180} = \frac{k\pi}{10800}$.
由此可得误差估计
\begin{equation}
    \begin{split}
        |\cos x - L'_1(x)| &= |\cos x - L_1(x) + L_1(x) - L'_1(x)| \\
        &\leq |\cos x - L_1(x)| + |L_1(x)-L'_1(x)|
    \end{split}
    \notag
\end{equation}
将误差估计分为两部分分别计算
\begin{equation}
    \begin{split}
        |\cos x - L_1(x)| &= \left| \frac{1}{2}(-\cos \xi)(x-x_k)(x-x_{k+1}) \right| \\
        &\leq \frac{1}{2} |(x-x_k)(x-x_{k+1})| \\
        &\leq \frac{1}{2}\times \left( \frac{1}{2} \times \frac{\pi}{10800} \right) ^2 \\
        &\approx 1.06 \times 10^{-8}
    \end{split}
    \notag
\end{equation}
\begin{equation}
    \begin{split}
        |L_1(x)-L'_1(x)| &= |e(f^*(x_k))|\frac{x_{k+1}-x}{x_{k+1}-x_k} + |e(f^*(x_{k+1}))|\frac{x-x_k}{x_{k+1}-x_k}\\
        &\leq \max \{|e(f^*(x_k))|, |e(f^*(x_{k+1}))|\} \left( \frac{x_{k+1}-x}{x_{k+1}-x_k} + \frac{x-x_k}{x_{k+1}-x_k} \right) \\
        &= \max \{|e(f^*(x_k))|, |e(f^*(x_{k+1}))|\}
    \end{split}
    \notag
\end{equation}
由有效数字的定义可得
\begin{equation}
    |e(f^*(x_k))| \leq \frac{1}{2}\times 10^{m_k-4} \notag
\end{equation}
所以有
\begin{equation}
    \max \{|e(f^*(x_k))|, |e(f^*(x_{k+1}))|\} \leq \max \left\{ \frac{1}{2}\times 10^{m_k-4}, \frac{1}{2}\times 10^{m_{k+1}-4} \right\} = \frac{1}{2}\times 10^{\max \{m_k, m_{k+1}\}-4} \notag
\end{equation}
综上所述
\begin{equation}
    |\cos x-L'_1(x)| \leq 1.06\times 10^{-8} + \frac{1}{2}\times 10^{\max \{m_k, m_{k+1}\}-4} \notag
\end{equation}
在区间 $[0, \frac{\pi}{2}]$上可得
\begin{equation}
    |\cos x-L'_1(x)|\leq 1.06 \times 10^{-8} + \frac{1}{2}\times 10^{-5} = 0.50106 \times 10^{-5} \notag
\end{equation}
\section*{6.}
\subsection*{i)}
\noindent
函数 $\sum_{j=0}^{n}x_j^kl_j(x)$为函数 $x^k$的Lagarnge插值多项式,同时 $x^k$也为自身的插值多项式,由于插值多项式具有唯一性,所以可知 $\sum_{j=0}^{n}x_j^kl_j(x) \equiv x^k$
\subsection*{ii)}
\noindent
\begin{equation}
    \begin{split}
        \sum_{j=0}^{n}(x_j-x)^kl_j(x) &= \sum_{j=0}^{n} \left[ \sum_{i=0}^{n}\tbinom{k}{i}x_j^i(-x)^{(k-i)i_j(x)} \right] \\
        &= \sum_{j=0}^{n} \sum_{i=0}^{k} \left[ \tbinom{k}{i}x_j^i(-x)^{k-i}i_j(x) \right]\\
        &= \sum_{i=0}^{k}\left[ \tbinom{k}{i} (-x)^{k-i} \sum_{j=0}^{n}x_j^i l_j(x) \right] \\
    \end{split}
    \notag
\end{equation}
由 i)得到的结论可知
\begin{equation}
    \sum_{j=0}^{n}(x_j-x)^kl_j(x)=\sum_{i=0}^{k} \tbinom{k}{i}(-x)^{k-i}x^i = (x-x)^k \equiv 0 \notag
\end{equation}


\section*{17.}


\end{document} 
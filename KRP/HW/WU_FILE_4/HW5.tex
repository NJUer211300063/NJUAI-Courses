\documentclass[12pt]{article}
% \usepackage[margin=1.25in]{geometry}
\usepackage[inner=2.0cm,outer=2.0cm,top=2.5cm,bottom=2.5cm]{geometry}
\usepackage{color}
\usepackage{ctex}
\usepackage{graphicx}
\usepackage{amssymb}
\usepackage{amsmath}
\usepackage{amsthm}
\usepackage{bm}
\usepackage{hyperref}
\usepackage{multirow}
\usepackage{mathtools}
\usepackage{enumerate}

\def \ALC {\mathcal{ALC}}
\def \K {\mathcal{K}}
\def \J {\mathcal{J}}
\def \T {\mathcal{T}}
\def \A {\mathcal{A}}
\def \I {\mathcal{I}}

\newcommand{\homework}[5]{
	\pagestyle{myheadings}
	\thispagestyle{plain}
	\newpage
	\setcounter{page}{1}
	\noindent
	\begin{center}
		\framebox{
			\vbox{\vspace{2mm}
				\hbox to 6.28in { {\bf KRP \hfill #2} }
				\vspace{6mm}
				\hbox to 6.28in { {\Large \hfill #1 \hfill} }
				\vspace{6mm}
				\hbox to 6.28in { {\it Instructor: {\rm #3} \hfill Name: {\rm #4}, StudentId: {\rm #5}}}
				\vspace{2mm}}
		}
	\end{center}
	% \markboth{#4 -- #1}{#4 -- #1}
	\vspace*{4mm}
}


\begin{document}
\large
	%==========================Put your name and id here==========================
	\homework{Homework 5}{Spring 2023}{YiZheng Zhao}{张运吉}{211300063}
    \paragraph{Question 1. CWA and OWA, querying without TBox}~{}
    \begin{enumerate}
        \item [(1)]
        \begin{align*}
            \text{DL notation }&: \text{ Album(Fantasy)} \\
            \text{FOL notation }&: \text{ Album(Fantasy)} \\
            \text{DL notation }&: \neg \text{StudioAlbum } \sqcup \neg \text{LiveAlbum(2004\_Incomparable\_Concert)} \\
            \text{FOL notation }&: \neg \text{StudioAlbum(2004\_Incomparable\_Concert)} \vee \\ &\neg \text{LiveAlbum(2004\_Incomparable\_Concert)} \\
            \text{DL notation }&: \exists\text{hasFriend.}\top\text{(Jay\_Chou)} \\
            \text{FOL notation }&: \exists \text{x.(hasFriend(Jay\_Chou, x)} \\
            \text{DL notation }&: \exists\text{hasFriend.}\exists\text{dancesWith.Song(Jay\_Chou)}\\
            \text{FOL notation }&: \exists \text{x.(hasFriend(Jay\_Chou, x)} \wedge \exists \text{y.(dancesWith(x, y)} \wedge \text{Song(y)))} \\
            \text{DL notation }&: \exists\text{hasFriend.\{Jay
            \_Chou\}(Vincent\_Fang)}\\
            \text{FOL notation }&: \text{hasFriend(Vincent\_Fang, Jay\_Chou)} 
        \end{align*}
        \item [(2)]
        (1).No (2).Yes (3).No (4).Yes (5).Yes (6).Yes (7).Yes (8).Yes (9).Yes (10).No \\
        (11).No (12).No (13).No (14).Yes (15).No (16).No (17).Yes (18).No (19).No \\ (20).No
        (21).Yes (22).Yes (23).No (24).No (25).Yes (26).Yes
        \item[(3)]
        (1).Don't know (2).Yes (3).Don't know (4).Don't know (5).Don't know \\ 
        (6).Don't know (7).Don't know (8).Yes (9).Yes (10).Don't know
        (11).Don't know \\
        (12).Don't know (13).Don't know (14).Yes (15).Don't know (16).Don't know \\
        (17).Yes (18).Don't know (19).Don't know (20).Don't know (21).  Don't know \\
        (22).Don't know (23).Don't know (24).Don't know (25).Don't know(26)Don't know
        \item [(4)]
        \textbf{For CWA}: \\
        answer(F1(x), Dmusic) = \{ Jay\_Chou, Eason\_Chan \} \\
        answer(F2(x), Dmusic) = \{ Will\_Liu, Black\_Cat, The\_Eight\_Dimensions, Secret, Together, Ta-yu\_Lo, Jolin\_Tsai, Vincent\_Fang, Hidden\_Track, Herbalist\_Manual, Jay, Common\_Jasmin\_Orange, Initial\_D, Elimination, Fantasy, Pearl\_of\_the\_Orient, 2004\_Incomparable\_Concert, Rewind \} \\
        answer(F3(x, y), Dmusic) = \{(Jay\_Chou, Vincent\_Fang), (Jay\_Chou, Will\_Liu)\} \\
        answer(F4(x), Dmusic) = \{ Vincent\_Fang \}
        \item[(5)]
        \textbf{For OWA}: \\
        certanswer(F1(x), Dmusic) = \{ Jay\_Chou, Eason\_Chan \} \\
        certanswer(F2(x), Dmusic) = $\emptyset$ \\
        certanswer(F3(x, y), Dmusic) = \{ (Jay\_Chou, Vincent\_Fang), (Jay\_Chou, Will\_Liu) \} \\
        certanswer(F4(x), Dmusic) = $\emptyset$
    \end{enumerate}


    \newpage
    \paragraph{Question 2. Querying with TBox}~{}
    \\

	\begin{enumerate}
        \item [(1)]
        (1).Yes (2).Yes (3).No (4).Yes (5).Don't know (6).Yes (7).Yes (8).Yes (9).Yes \\
        (10).Don't know (11).Don't know (12).No (13).Yes (14).Yes (15).Don't know \\
        (16).Yes (17).Yes (18).Don't know (19).No (20).Don't know (21).Don't know \\
        (22).Don't know (23).Don't know (24).Don't know (25).Don't know
        (26)Don't know
        \item [(2)]
        Song(Herbalist\_Manual) \\
        Song(Elimination) \\
        Singer(Jay\_Chou) \\
        Lyricist(Vincent\_Fang) \\
    \end{enumerate}

    \newpage
    \paragraph{Question 3. Computing $\mathcal{I}_{\mathcal{T}, \mathcal{A}}$ in $\mathcal{EL}$ Consider the $\mathcal{EL}$ TBox $\mathcal{T}$ :}~{}
    \\

    \begin{enumerate}
        \item [(1)]
        We initialize $\boldmath{S}$ and $\boldmath{R}$ as follow: \par
        $\boldsymbol{S}(Monster)=\{Captain, LeadGuitarist \}$ \par
        $\boldsymbol{S}(Ashin)=\{Vocalist\}$ \par
        $\boldsymbol{S}(Masa)=\{Bassist\}$ \par
        $\boldsymbol{S}(Ming)=\{Drummer\}$ \par
        $\boldsymbol{S}(Stone)=\{RhythmGuitarist\}$ \par
        $\boldsymbol{S}(Marday)=\{Band\}$ \par
        $\boldsymbol{S}(Amuse)=\emptyset$ \par
        $\boldsymbol{S}\left(d_{A}\right)=\{A\} \text { for all } d_{A} \in \Delta^{\mathcal{I}_{\T, \A}}$ \par
        $\boldsymbol{R}(managed\_by)=\{(Marday, Amuse)\}$ \par
        $\boldsymbol{R}(captained\_by)=\emptyset$ \par
        $\boldsymbol{R}(plays\_for)=\emptyset$ \par
        Update $\boldsymbol{S}$ and $\boldsymbol{R}$ using simpleR, conjR, rightR and leftR, we get: \par
        $\boldsymbol{S}(Monster) = \{Captain, LeadGuitarist, Guitarist\}$ \\
        $\boldsymbol{S}(Stone) = \{RhythmGuitarist, Guitarist\}$ \\
        $\boldsymbol{S}(d_{LeadGuitarist}) = \{d_{LeadGuitarist}, d_{Guitarist}\}$ \\
        $\boldsymbol{S}(d_{RhythmGuitarist}) = \{d_{RhythmGuitarist}, d_{Guitarist}\}$ \\
        $\boldsymbol{R}(managed\_by) = \{(Marday, Amuse), (Marday, d_{Manager}), (d_{Band}, d_{Manager}), \\ (d_{Manager}, d_{Manager})\}$ \\
        $\boldsymbol{R}(plays\_for) = \{(Ashin, d_{Band}), (Monster, d_{Band}), (Masa, d_{Band}), (Stone, d_{Band}), \\ (Ming, d_{Band}),
        (d_{Guitarist}, d_{Band}), (d_{Bassist}, d_{Band}), (d_{Vocalist}, d_{Band}), (d_{Drummer}, d_{Band})\}$
        $\boldsymbol{R}(captained\_by) = \{(Mayday, d_{Captain}), (d_{Band}, d_{Captain})\}$ \\
        The intepretation $\I_{\T, \A}$ is as follow: 
        \begin{align*}
            \Delta^{\mathcal{I}} &= \{Monster, Ashin, Masa, Stone, Ming, Mayday, Amuse, d_{Vocalist}, d_{Guitarist},  \\ & d_{Bassist}, d_{Drummer}, d_{LeadGuitarist}, d_{RhythmGuitarist}, d_{Band}, d_{Manager}, d_{Captain}\}
        \end{align*}
        $Guitarist^{\I_{\T, \A}}= \{Monster, Stone, d_{LeadGuitarist}, d_{RhyrhmGuitarist}, d_{Guitarist}\}$ \\
        $Bassist^{\I_{\T, \A}}= \{Masa, d_{Bassist}\}$ \\
        $Drummer^{\I_{\T, \A}}= \{Ming, d_{Drummer}\}$ \\
        $Vocalist^{\I_{\T, \A}} = \{Ashin, d_{Vocalist}\}$ \\
        $LeadGuitarist^{\I_{\T, \A}} = \{Monster, d_{Leadguitarist}\}$ \\
        $RhythmGuitarist^{\I_{\T, \A}}= \{Stone, d_{RhythmGuitarist}\}$ \\
        $Band^{\I_{\T, \A}}= \{Marday, d_{Band}\}$ \\
        $Manager^{\I_{\T, \A}}= \{d_{Manager}\}$ \\
        $Captain^{\I_{\T, \A}}= \{Monster, d_{Captain}\}$ \\
        $managed\_by^{\I_{\T, \A}}= \{(Marday, Asume), (Marday, d_{Manager}, (d_{Band}, d_{Manager}), \\ (d_{Manager}, d_{Manager})\}$ \\
        $captained\_by^{\I_{\T, \A}}= \{(Marday, d_{Captain}), (d_{Band}, d_{Captain})\}$ \\
        $plays\_for^{\I_{\T, \A}}= \{(Monster, d_{Band}), (Masa, d_{Band}), (Ashin, d_{Band}), (Stone, d_{Band}), \\ (Ming, d_{Band}), (d_{Guitarist}, d_{Band}), (d_{Bassist}, d_{Band}), (d_{Vocalist}, d_{Band}), (d_{Drummer}, d_{Band})\}$
        \item[(2)]
        \begin{enumerate}[i.]
            \item Yes. \\
            Because $(Ashin, d_{Band}) \in plays\_for^{\I_{\T, \A}}$ and $d_{Band} \in Band^{\I_{\T, \A}}$.
            \item No. 
            \item Yes. \\
            Because $(Monster, d_{Band}) \in plays\_for^{\I_{\T, \A}}, (d_{Band}, d_{Captain}) \in captained\_by^{\I_{\T, \A}}$ and $d_{Captain} \in Captain^{\I_{\T, \A}}$. 
            \item Yes. \\
            Because $(Ming, d_{Band}) \in plays\_for^{\I_{\T, \A}}, (d_{Band}, d_{Manager}) \in managed\_by^{\I_{\T, \A}}$ and $d_{Manager} \in Manager^{\I_{\T, \A}}$. 
        \end{enumerate}
        \item[(3)] 
        \begin{enumerate}[i.]
            \item $certainswer(F(x, y), (\T , \A)) = \{Monster, Ashine, Ming, Stone, Masa\} \times \{Monster, Ashine, Ming, Stone, Masa\}$ \\
            $answer(F(x, y), \I_{\T, \A}) = (Guitarist^{\I_{\T, \A}} \cup Vocalist^{\I_{\T, \A}}\cup Drummer^{\I_{\T, \A}}\cup Bassist^{\I_{\T, \A}}) \times (Guitarist^{\I_{\T, \A}} \cup Vocalist^{\I_{\T, \A}} \cup Drummer^{\I_{\T, \A}}\cup Bassist^{\I_{\T, \A}}) $ ($"\times"$ is cartesian product)
            \item $certainswer(F, (\T , \A)) = ”No”$. \\
            $answer(F, \I_{\T, \A}) = ”Yes”$.
        \end{enumerate}
    \end{enumerate}




    \newpage
    \paragraph{Question 4. SQL and Conjunctive Queries}~{}
    \\
    \begin{enumerate}
        \item [(1)]
        \begin{align*}
            \text{ID}^{\mathcal{I}_\mathcal{D}} &= \{2101, 2102, 2103, 2104, 30000150, 30000160, 30000170\} \\
            \text{Name}^{\mathcal{I}_{\mathcal{D}}} &= \{ \text{Jay\_Chou, Jolin\_Tsai, Stefanie\_Sun, Ta-yu\_Lo} \} \\
            \text{StudentID}^{\mathcal{I}_{\mathcal{D}}} &= \{ 2101, 2102, 2103, 2104 \} \\
            \text{Since}^{\mathcal{I}_{\mathcal{D}}} &= \{ 2020, 2021 \} \\
            \text{CourseID}^{\mathcal{I}_{\mathcal{D}}} &= \{ 30000150, 30000160, 30000170 \} \\
            \text{Title}^{\mathcal{I}_{\mathcal{D}}} &= \{ ML, KRP, NLP \} \\
            \text{Person}^{\mathcal{I}_{\mathcal{D}}} &= \{ (2101, Jay\_Chou), (2102, Jolin\_Tsai), \\& (2103, Stefanie\_Sun), (2104, Ta-yu\_Lo) \} \\
            \text{Enrollment}^{\mathcal{I}_{\mathcal{D}}} &= \{ (2102, 2020), (2103, 2021), (2104, 2020) \} \\
            \text{Attendance}^{\mathcal{I}_{\mathcal{D}}} &= \{ (2101, 30000160), \\ &(2102, 30000160), (2102, 30000170), (2103, 30000150) \} \\
            \text{Course}^{\mathcal{I}_{\mathcal{D}}} &= \{ (30000150, ML), (30000160, KRP), (30000170, NLP) \}
        \end{align*}
        \item [(2)]
        $F_a(x, y) = Person(x, y)$ \\
        $F_b(x) = \exists y.\exists z.(Person(y, x) \wedge Attendance(y, z) \wedge Course(z, KRP))$ \\
        $F_c(x) = \exists y.\exists z.(Person(y, x) \wedge Enrollment(y, z) \wedge \forall c.\neg attendance(y, c))$

$F_a(x, y)$ and $F_b(x)$ are conjunctive queries but $F_c(x)$ is not conjunctive query.
        \item [(3)]
        Answer $Q$ in the context of $\mathcal{D}$: \\
answer(Qa, D) = \{ (2101, Jay\_Chou), (2102, Jolin\_Tsai), (2103, Stefanie\_Sun), (2104, Ta-yu\_Lo) \} \\
answer(Qb, D) = \{ Jay\_Chou, Jolin\_Tsai \} \\
answer(Qc, D) = \{ Ta-yu\_Lo \} \\
Answer $f_Q$ in the context of $\I_\mathcal{D}$: \\
answer(Fa, $\I_\mathcal{D}$) = \{ (2101, Jay\_Chou), (2102, Jolin\_Tsai), (2103, Stefanie\_Sun), (2104, Ta-yu\_Lo) \} \\
answer(Fb, $\I_\mathcal{D}$) = \{ Jay\_Chou, Jolin\_Tsai \} \\
answer(Fc, $\I_\mathcal{D}$) = \{ Ta-yu\_Lo \}
    \end{enumerate}

    \newpage
    \paragraph{Question 5. Certain Answers in Diferent Contexts}~{}
    \\
    \begin{enumerate}
        \item [(1)]
        1. $\emptyset$  2. $\emptyset$  3. "No"  4. $\emptyset$
        \item [(2)]
        1. \{ (Jolin\_Tsai, Stefanie\_Sun), (Stefanie\_Sun, Jay\_Chou), (Stefanie\_Sun, Stefanie\_Sun) \} \\
        2. \{ Jay\_Chou, Stefanie\_Sun \} \\
        3. "Yes" \\
        4. \{ (Jay\_Chou, Jolin\_Tsai), (Stefanie\_Sun, Jay\_Chou), (Jolin\_Tsai, Jolin\_Tsai), (Jolin\_Tsai, Stefanie\_Sun), (Stefanie\_Sun, Stefanie\_Sun) \}
    \end{enumerate}

    \newpage
    \paragraph{Question 6. Reasoning in $\mathcal{EL}$}~{}
    \\

    We initialize $\boldsymbol{S}$ and $\boldsymbol{R}$ :\\
    $$
    \begin{array}{ll}
    \boldsymbol{S}(a)=\emptyset \\ \boldsymbol{S}(b)=\{X\} \\
    \boldsymbol{S}\left(d_X\right)=\{X\} \\ \boldsymbol{S}\left(d_Y\right)=\{Y\} \\
    \boldsymbol{R}(r)=\{(a, b)\} &
    \end{array}
    $$ \par
    Update $\boldsymbol{R}(r)$ using rightR: 
    $$
    \boldsymbol{R}(r)=\left\{(a, b),\left(b, d_Y\right),\left(d_X, d_Y\right)\right\}
    $$ \par
    Again, update $\boldsymbol{R}(r)$ using rightR:
    $$
    \boldsymbol{R}(r)=\left\{(a, b),\left(b, d_Y\right),\left(d_X, d_Y\right),\left(d_Y, d_Y\right)\right\}
    $$ \par
    We can get the interpretation $\mathcal{I}_{\mathcal{T}, \mathcal{A}}$ as follow:
    $$
    \begin{aligned}
    & \Delta^{\mathcal{I}_{\mathcal{T}, \mathcal{A}}}=\left\{a, b, d_X, d_Y\right\} \\
    & X^{\mathcal{I}_{\mathcal{T}, \mathcal{A}}}=\{b, d_X\} \\
    & Y^{\mathcal{I}_{\mathcal{T}, \mathcal{A}}}=\left\{d_Y\right\} \\
    & r^{\mathcal{I}_{\mathcal{T}, \mathcal{A}}}=\left\{(a, b),\left(b, d_Y\right),\left(d_X, d_Y\right),\left(d_Y, d_Y\right)\right\}
    \end{aligned}
    $$


    \newpage
    \paragraph{Question 7. Ontology-Mediated Query Answering}~{}
    \\

    \textbf{Don' t know}.
    \par
    The answer is not "No" because there is a model of $(\T, \A)$ in which clash is non-empty. See as follow: \par
    $$
    \begin{aligned}
    & \Delta^{\mathcal{I}}=\{0,1,2,3\}, \\
    & r^{\mathcal{I}}=\{(0,1),(1,2),(2,3),(3,0)\} \\
    & \text { green }{ }^{\mathcal{I}}=\{0,1,2\} \\
    & \text { red }^{\mathcal{I}}=\{3\} \\
    & \text { clash }^{\mathcal{I}}=\{2, 3\}
    \end{aligned}
    $$ \par
    The answer is not "Yes" because there is a model of $(\T, \A)$ in which clash is empty. See as follow: \par
    $$
    \begin{aligned}
    & \Delta^{\mathcal{I}}=\{0,1,2,3\}, \\
    & r^{\mathcal{I}}=\{(0,1),(1,2),(2,3),(3,0)\} \\
    & \text { green }{ }^{\mathcal{I}}=\{0,1,2,3\} \\
    & \text { red }^{\mathcal{I}}=\emptyset \\
    & \text { clash }^{\mathcal{I}}=\emptyset
    \end{aligned}
    $$ \par
    Therefore, the answer is "Don' t know".
    

    \newpage
    \paragraph{Question 8. Σ-Reducts in Acyclic $\mathcal{EL}$}~{}
    \\
    \begin{enumerate}
        \item [(1)]    \textbf{Yes}, there always exists a $\Sigma$-reduct $\mathcal{T}_{1}$ for any given acyclic $\mathcal{EL}$ ontology $\mathcal{T}_{2}$ and any $\Sigma \subseteq \operatorname{sig}\left(\mathcal{T}_{2}\right)$
        \item [(2)]
        \textbf{Algorithm}: \\
        Compute\_$\Sigma$-reduct($\mathcal{T}_{2}$, $\Sigma$) : \\
        1. Let $\mathcal{T}_{1} = \emptyset$. \\
        2. For all concept axioms and role axioms, if its signature are subset of $\Sigma$ to $\mathcal{T}_{1}$ then add it to $\mathcal{T}_{1}$. \\
        3. For each  $C \sqsubseteq$ D in $\mathcal{T}_{1}$: If $\mathcal{T}_{2} \not \models C \sqsubseteq D$, remove the axiom $C \sqsubseteq D$ from $\mathcal{T}_{1}$, otherwise do nothing. \\
        4. return $\mathcal{T}_{1}$.  \\
        \textbf{Termination}:\\
        The algorithm terminates because there is a finite number of concept and role axioms in $\mathcal{T}_2$, and at each iteration, either an axiom is added to $\mathcal{T}_1$ or removed from $\mathcal{T}_1$. Since the number of axioms in $\mathcal{T}_1$ is nonincreasing, the algorithm eventually terminates.
        \\
        \textbf{Soundness}:\\
        The algorithm only adds concept and role axioms from $\mathcal{T}_2$ whose signatures are subsets of $\Sigma$ to $\mathcal{T}_1$. Therefore, the resulting $\Sigma$-reduct $\mathcal{T}_1$ satisfies the condition $\operatorname{sig}(\mathcal{T}_1) \subseteq \Sigma$. Additionally, for each concept subsumption axiom $C \sqsubseteq D$ in $\mathcal{T}_1$, if $\mathcal{T}_1 \models C \sqsubseteq D$, then $\mathcal{T}_2 \models C \sqsubseteq D$ by the construction of $\mathcal{T}_1$.
        \\
        \textbf{Completeness}:\\
        The algorithm iteratively checks each concept subsumption axiom $C \sqsubseteq D$ in $\mathcal{T}_1$ and removes it if $\mathcal{T}_2 \not\models C \sqsubseteq D$. Therefore, if $\mathcal{T}_1 \models C \sqsubseteq D$, then $\mathcal{T}_2 \models C \sqsubseteq D$. Thus, the algorithm is complete.
    \end{enumerate}


    \newpage
    \paragraph{Question 9 (with 1 bonus mark). Simpleness of ABox}~{}
    \\

    This \textbf{ doesn't affect }the data complexity results because the lower bound and upper bound do not change. 

    Lower bound:

    - $\mathcal{ALC}$: coNP-hard \par
    - $\mathcal{EL}$: P-hard \par
    - $\text{DL-Lite}$: AC$^0$ \par

    Because questions like non-3-colorability and path system accessibility can be still reduced into $\mathcal{ALC}$ and $\mathcal{EL}$ CQ-entailment problem by the same way with simple ABoxes.

    This doesn't affect the upper bound of the data complexity results, too.

    Upper bound:

    - $\mathcal{ALC}$: coNP-complete, because we can still use tableau algorithm to solve the problem. \par
    - $\mathcal{EL}$: P-complete, because we can reduce it to Datalog query entailment with PTime data complexity. \par
    - $\text{DL-Lite}$: AC$^0$, because we can reduce it to entailment of their FO-rewriting $q_{\mathcal{T}}$. \par

    So this doesn't affect the data complexity result.

    \newpage
    \paragraph{Question 10 (with 1 bonus mark). $K$-Colorability}~{}
    \\

    \textbf{Yes, we can. } \par

    We can consider the following $\mathcal{ALC}$ TBox and Boolean CQ:

    $\mathcal{T} = \{ \top \sqsubseteq \underset{C \in \text{colors}}{\Large\sqcup}C, C \sqcap \exists r.C \sqsubseteq D \text{ for } C \in \text{colors} \}$

    $q = \exists x.D(x)$

    And we translated the input graph $\boldsymbol{G} = (V, E)$ into the ABox:

    $\mathcal{A} = \{ (u, v):r | \{ u, v \} \in E \}$

    Then it is equal to prove that $(\mathcal{T}, \mathcal{A}) \models q$ iff $\mathcal{A}$ is not k-colourable.

    Because 3-colorability is NP-hard, thus k-colorability is NP-hard too. So we can know that the query entailment problem for conjunctive queries is coNP-hard w.r.t data complexity in $\mathcal{ALC}$.

    \textbf{But if we let $k$ is fixed, the result depends on the value of $k$}.
    
    \begin{itemize}
        \item k = 1. \\
        $\boldsymbol{G}$ is 1-colorable iff there's no edge between any nodes. This can be checked in PTime.
        \item k = 2. \\
        $\boldsymbol{G}$ is 2-colorable iff $\boldsymbol{G}$ can be devided into two disjoint sets, i.e. $\boldsymbol{G}$ is a bipartite graph. This can also be checked in PTime.
        \item k = 3. \\
        3-colorability in $\boldsymbol{G}$ is NP-hard and it can prove that the query entailment problem for conjunctive queries is coNP-hard w.r.t data complexity in $\mathcal{ALC}$.

    \end{itemize}

\end{document}

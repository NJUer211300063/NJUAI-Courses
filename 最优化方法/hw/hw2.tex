\documentclass[a4paper]{article}
% \usepackage[margin=1.25in]{geometry}
\usepackage[inner=2.0cm,outer=2.0cm,top=2.5cm,bottom=2.5cm]{geometry}
% \usepackage{ctex}
\usepackage{color}
\usepackage{graphicx}
\usepackage{amssymb}
\usepackage{amsmath}
\usepackage{amsthm}
\usepackage{bm}
\usepackage{hyperref}
\usepackage{multirow}
\usepackage{mathtools}
\usepackage{enumerate}
\usepackage{braket}
\usepackage[]{ctex}

\newcommand{\homework}[5]{
    \pagestyle{myheadings}
    \thispagestyle{plain}
    \newpage
    \setcounter{page}{1}
    \noindent
    \begin{center}
    \framebox{
        \vbox{\vspace{2mm}
        \hbox to 6.28in { {\bf Optimization Methods \hfill #2} }
        \vspace{6mm}
        \hbox to 6.28in { {\Large \hfill #1 \hfill} }
        \vspace{6mm}
        \hbox to 6.28in { {\it Instructor: {\rm #3} \hfill Name: {\rm #4}, StudentId: {\rm #5}}}
        \vspace{2mm}}
    }
    \end{center}
    \vspace*{4mm}
}


\newenvironment{solution}
{\color{blue} \paragraph{Solution.}}
{\newline \qed}

\begin{document}
%- Write your name and id here
\homework{Homework 2}{Fall 2022}{Lijun Zhang}{张运吉}{211300063}

\paragraph{Notice}
\begin{itemize}
    \item The submission email is: \href{mailto:optfall2022@163.com}{optfall2022@163.com}.
    \item Please use the provided \LaTeX{} file as a template.
    \item If you are not familiar with \LaTeX{}, you can also use Word to generate a \textbf{PDF} file.
\end{itemize}

\paragraph{Problem 1: Convexity}
~\\

\begin{enumerate}[a)]
    
    \item Suppose $f$ and $g$ are both convex, nondecreasing (or nonincreasing), and positive real-valued functions defined on $\mathbb{R}$, prove that $fg$ is convex on $\textbf{dom}(f)\cap \textbf{dom}(g)$.
    
	\item Let~$f$~be twice differentiable, with~$\textbf{dom}(f)$~convex. Prove that~$f$~is convex if and only if
    \begin{equation}
    \left( \nabla f(x) - \nabla f(y) \right)^{\top} (x - y) \geqslant 0, \notag
    \end{equation}
    for all x, y.

    \item Let~$f : \mathbb{R}^n \rightarrow \mathbb{R}$~be a function. Its perspective transform~$g : \mathbb{R}^{n+1} \rightarrow \mathbb{R}$~is defined by
    \begin{equation}
    g(x, t) = t f\left(\frac{x}{t}\right), \notag
    \end{equation}
    with domain~$\textbf{dom}(g) = \{(x, t) \in \mathbb{R}^{n+1} : x \in \textbf{dom}(f), t > 0\}$. Use the definition of convexity to prove that if~$f$~is convex, then so is its perspective transform~$g$.
\end{enumerate}

\begin{solution} 
    \subsection*{(a)}
    $\because f,g$ 是凸函数,$\therefore dom(f), dom(g)$是凸集合,
    $\therefore dom(f)\cap dom(g)$是凸集合。

    $\forall x,y \in dom(f\cap g), \forall \theta \in [0,1]$, $g(\theta x+(1-\theta)y) \leq \theta g(x)+(1-\theta)g(y)$

    再用$f$作用于不等式两边,并且由于$f$是非递减的,
    
    得到$f[g(\theta x+(1-\theta)y)] \leq f[\theta g(x)+(1-\theta)g(y)] \leq \theta fg(x) + (1-\theta)fg(x)$,
    
    最后一个不等式是由于$f$是凸函数。

    $\therefore fg$是凸函数。

    \subsection*{(b)}
    凸函数的一阶条件:
    $f(x)\geq f(y)+\nabla f(y)^{\top}(x-y) ...(1)$ 
    
    $f(y)\geq f(x)+\nabla f(x)^{\top}(y-x) ... (2)$,
    $ \forall x,y \in dom(f)$
    
    $(\nabla f(x)-\nabla f(y))^{\top}(x-y)\geq 0 \iff$ $\nabla f(x)^{\top}(x-y) - \nabla f(y)^{\top}(x-y) \geq 0$

    $\iff \nabla f(x)^{\top}(x-y)-(f(x)-f(y)) \geq 0$由(1),

    $\iff f(y)-[f(x)+\nabla f(x)^{\top}(y-x)] \geq 0$
    
    由(2),上式成立。

    \subsection*{(c)}
    因为$dom(f)$是凸集合,易知$\textbf{dom}(g) = \{(x, t) \in \mathbb{R}^{n+1} : x \in \textbf{dom}(f), t > 0\}$也是凸集合。

    $\forall (x,t),(y,t) \in dom(g)$, 

    \begin{aligned}
        &\quad g(\theta(x,t)+(1-\theta)(y,t)) \\
        &=g((\theta x, \theta t)+((1-\theta) y, t(1-\theta))) \\
        &=g(\theta x+(1-\theta)y, t) \\
        &= tf(\frac{\theta x+(1-\theta y)}{t}) \\ 
        &= tf(\theta \frac{x}{t}+(1-\theta)\frac{y}{t}) \\
        &\leq t\theta f(\frac{x}{t})+t(1-\theta)f(\frac{y}{t}) \\
        &= \theta g(x,t) +(1-\theta)g(y,t) 
    \end{aligned}
    
    $\therefore g$是凸函数。
\end{solution}


\paragraph{Problem 2: Convex Functions}
~\\

\noindent
\begin{enumerate}[a)]

    \item Suppose $f_i:\mathbb{R}^n \mapsto \mathbb{R}$ are convex for all $i\in\{1,2,\cdots,n\}$. Prove that the function~$f:\mathbb{R}^n \mapsto \mathbb{R}$, defined as
    \begin{equation}
    f(x) = \sum^n_{i=1} e^{-1/f_i(x)}, \notag
    \end{equation}
    is convex on $\textbf{dom}(f)=\{x\in\mathbb{R}^n \vert f_i(x)<0,i=1,2,\cdots,n\}$.
    
    \item Show that the logarithmic barrier function for the second-order cone, defined as
    \begin{equation}
        f(x,t) = -\log(t^2-x^{\top}x) \notag
    \end{equation}
    is convex on $\textbf{dom}(f)=\{(x,t)\in\mathbb{R}^n\times\mathbb{R} | t>\Vert x\Vert_2\}$. (\textit{Hint: consider the function $-\log(t-(1/t)u^{\top}u)$})
    
    \item Suppose $A\in\mathbb{R}^{n\times n}, b\in\mathbb{R}^{n}$. Show that the function
    \begin{equation}
        f(x)=\frac{\Vert Ax-b\Vert^2_2}{1-x^{\top}x} \notag
    \end{equation}
    is convex on $\textbf{dom}(f)=\{x\in\mathbb{R}^n|\Vert x\Vert_2\le 1\}$.

\end{enumerate}

\newpage
\begin{solution}
\subsection*{(a)}
记$h(z)=-\frac{1}{z}, dom(h)=\{z<0\}$,$h(z)$是凸函数且非递减的,$f_i(x)<0$且是凸函数,由p1(1)的结论
知$h(f_i(x))=-\frac{1}{f_i(x)}$是凸函数,又因为$e^x$是凸函数且非递减的,同理可知$e^{-\frac{1}{f_i(x)}}$
是凸函数,由于非负加权求和保持凸性,所以$f(x) = \sum^n_{i=1} e^{-1/f_i(x)}$是凸函数。
\subsection*{(b)}
$f(x,t) = -\log(t^2-x^{\top}x)=-\log(t-\frac{1}{t}x^{\top}x)-\log t$

设$g(x,t)=t-\frac{1}{t}x^{\top}x$,$h(x)=1-x^{\top}x$,

则$g(x,t)=th(\frac{x}{t})$

易知$h$是凹函数,所以$g$是凹函数。(透视运算保持凸性)

$\log g(x,t)$是凹函数。(复合保持凸性)

$\therefore f = -\log g(x,t)-\log t$是凸函数。

\subsection*{(c)}
先证明$h(x)=\frac{\Vert Ax-b\Vert^2_2}{c^{\top}x+d}$是凸函数,

$h(x)=g(y,t)=\frac{y^{\top}y}{t}$,令$(y,t)=(Ax-b,c^{\top}x+d)$

可以看出$h$是$g$经过仿射变换得到

而$g$是函数$x^{\top}x$的透视函数,由于$x^{\top}x$是图的,所以$g$是凸函数

$\therefore h$是凸的

最后在$h$中令$c=-x, d=1$可得$f(x)=\frac{\Vert Ax-b\Vert^2_2}{1-x^{\top}x}$是凸的
\end{solution}


\paragraph{Problem 3: Concave Function}
~\\

\noindent
Suppose~$0<p\le 1$. Show that the function
\begin{equation}
f(x) = \left( \sum^n_{i=1} x^p_i \right)^{\frac{1}{p}} \notag
\end{equation}
with~$\textbf{dom}(f) = \mathbb{R}^n_{+}$~is concave.

\begin{solution}
Write your answer here.

$\frac{\partial f(X)}{\partial x_i}=(\sum^n_{i=1} x^p_i)^{\frac{1}{p}-1}x_i^{p-1}=(\frac{f(x)}{x_i})^{1-p}$

$\frac{\partial ^2 f(x)}{\partial x_i \partial x_j}
= (1-p)(\frac{f(x)}{x_i})^{-p}(\frac{f(x)}{x_i})^{'}
= (1-p)(\frac{f(x)}{x_i})^{-p}\frac{1}{x_i}\frac{\partial f(x)}{\partial x_j}
= \frac{1-p}{f(x)}(\frac{f(x)^2}{x_ix_j})^{1-p}$

$\forall y \in \mathbb{R}^n_{+}, y^{\top}(\nabla ^2 f(x))y
= \frac{1-p}{f(x)}((\sum^n_{i=1} \frac{y_i f(x)^{1-p}}{x_i^{1-p}})^2-\sum^n_{i=1} \frac{y_i^2 f(x)^{2-p}}{x_i^{2-p}})$

由柯西不等式$|a|\cdot |b| \geq |a\cdotb|$,

令$a_i=(\frac{f(x)}{x_i})^{-\frac{p}{2}}, b_i=y_i(\frac{f(x)}{x_i})^{1-\frac{p}{2}}$

得$(\sum^n_{i=1} \frac{y_i f(x)^{1-p}}{x_i^{1-p}})^2 \leq \sum^n_{i=1} \frac{y_i^2 f(x)^{2-p}}{x_i^{2-p}}$

$\therefore \nabla ^2 f(x) \preceq 0$,即$f$是凹函数。
\end{solution}


\paragraph{Problem 4: Convexity and Conjugate Function}
~\\

\noindent
Let~$R:\Omega \mapsto \mathbb{R}$~be a strictly convex and continuously differentiable function defined on a closed convex set $\Omega$. Denote by $\Delta_{R} (x, y)$ the \textit{Bregman divergence} with respect to the function $R$, defined as
\begin{equation}\label{bregman divergence}
\Delta_{R} (x, y) = R(x) - R(y) - \langle \nabla R(y), x - y \rangle, ~~\forall x, y \in \Omega.
\end{equation}
That is, the difference between the value of $R$ at $x$ and the first order Taylor expansion of $R$ around $y$ evaluated at point $x$.

\begin{enumerate}[a)]
    \item Derive the \textit{Bregman divergence} of $f(x)=\sum_{i=1}^n x_i\log(x_i)$ with $\Omega=\mathbb{R}^n_{+}$.
    
    \item Let~$L$~be a convex and differentiable function defined on~$\Omega$~and~$C \subset \Omega$~be a convex set. 
    Let~$x_0 \in \Omega - C$ and define
    \begin{equation}
        x^* = \mathop{\arg\min}\limits_{x \in C} L(x) + \Delta_{R} (x, x_0). \notag
    \end{equation}
    Prove that for any~$y \in C$,
    \begin{equation}
        L(y) + \Delta_{R}(y, x_0) \geqslant L(x^*) + \Delta_{R} (x^*, x_0) + \Delta_{R} (y, x^*). \notag
    \end{equation}
    
    \item Recall that the definition of conjugate function is
    \begin{equation}
        f^{\star}(y) = \sup_{x \in \textbf{dom}(f)} \left(y^{\top}x - f(x)\right). \notag
    \end{equation}
    Let $R(x)=\frac{1}{2}x^{\top}Qx$ defined on $\mathbb{R}^n$, where $Q\in\mathbb{S}^n_{++}$. Derive the conjugates $R^{\star}$ and $R^{\star\star}$. Verify that $(\nabla R^{\star})(\nabla R(x))=x$ and $\Delta_{R}(x,y) = \Delta_{R^{\star}} (\nabla R(y), \nabla R(x))$.
\end{enumerate}

\begin{solution}
% Write your answer here.
\subsection*{(a)}
$\Delta_{f} (x, y)=f(x)-f(y)- \langle \nabla f(x), x-y\rangle$

$=\sum^n_{i=1} x_i\log x_i -\sum^n_{i=1} y_i\log y_i- \sum^n_{i=1} (x_i-y_i)(\log y_i+1)$

$=\sum^n_{i=1}(x_i\log \frac{x_i}{y_i}-(x_i-y_i))$
\subsection*{(b)}
要证 $\displaystyle L(y)+\Delta_{f}(y,x_0)\geqslant L(x^{*})+\Delta_{f}(x^{*},x_0)+\Delta_{f}(y,x^{*})$

即证 $\displaystyle L(y)+f(y)-f(x_0)-\nabla f(x_0)^{T}(y-x_0)\geqslant L(x^{*})+f(x^{*})-f(x_0)-\nabla f(x_0)^{T}(x^{*}-x_0)+f(y)-f(x^{*})-\nabla f(x^{*})^{T}(y-x^{*})$

即证 $\displaystyle L(y)\geqslant L(x^{*})+[\nabla f(x_0)-\nabla f(x^{*})]^{T}(y-x^{*})$

由 $L(y)$ 是凸函数可知 $L(y)\geqslant L(x^{*})+\nabla L(x^{*})^{T}(y-x^{*})$

令 $f(x)=L(x)+\Delta_{f}(x,x_0)=L(x)+f(x)-f(x_0)-\nabla f(x_0)^{T}(x-x_0)$, 因为其是数个凸函数相加, 结果仍然是凸函数

求梯度得 $\nabla f(x)=\nabla L(x)+\nabla f(x)-\nabla f(x_0)$, 因为在 $x=x^{*}$ 处取得最小值, 因此有 $\nabla f(x^{*})=\nabla L(x^{*})+\nabla f(x^{*})-\nabla f(x_0)=0$

因此 $\nabla L(x^{*})=\nabla f(x_0)-\nabla f(x^{*})$, 带入 $L(y)\geqslant L(x^{*})+\nabla L(x^{*})^{T}(y-x^{*})$

可知 $L(y)\geqslant L(x^{*})+[\nabla f(x_0)-\nabla f(x^{*})]^{T}(y-x^{*})$ 成立

因此原式成立.
\subsection*{(c)}
$R^{\star}(y) = \sup_{x \in \textbf{dom}(R)} \left(y^{\top}x - R(x)\right)$

令$h(x)=y^{\top}x-R(x)$, $\nabla h(x)=y-Qx$

令$\nabla h(x)=y-Qx=0$,得$x=Q^{-1}y$(也即$h(x)$的最大值点)

$\therefore R^{\star}(y) = y^{\top}Q^{-1}y-\frac{1}{2}({Q^{-1}y})^{\top}Q{Q^{-1}y}=\frac{1}{2}y^{\top}Q^{-1}y$

$\because \nabla ^2 R(x)=Q$是正定矩阵

$\therefore R$是凸函数

$\because$凸函数的共轭函数的共轭函数是其本身

$\therefore R^{\star \star}(x)=R(x)=\frac{1}{2}x^{\top}Qx$

$(\nabla R^{\star})(\nabla R(x))=Q^{-1}(Qx)=x$

$\Delta_{R^{\star}}(\nabla R(y), \nabla R(x))=R^{\star}(\nabla R(y))-R^{\star}(R(x))-\Braket{ \nabla R^{\star}(\nabla R(x)), \nabla R(y)-\nabla R(x)}$

$=\frac{1}{2}y^{\top}Qy-\frac{1}{2}x^{\top}Qx-x^{\top}Q(y-x)=\frac{1}{2}x^{\top}Qx-\frac{1}{2}y^{\top}Qy+(y^{\top}-x^{\top})Qy$

$\Delta_{R}(x,y)=\frac{1}{2}x^{\top}Qx-\frac{1}{2}y^{\top}Qy-Qy(x-y)$

$=\frac{1}{2}x^{\top}Qx-\frac{1}{2}y^{\top}Qy+(y^{\top}-x^{\top})Qy$

$\therefore \Delta_{R}(x,y)=\Delta_{R^{\star}}(\nabla R(y), \nabla R(x))$
\end{solution}


\paragraph{Problem 5: Projection}
~\\

\noindent
For any point~$y$, the projection onto a nonempty and closed convex set~$\mathcal{X}$~is defined as
\begin{equation}
\Pi_{\mathcal{X}} (y) = \mathop{\arg\min}\limits_{x \in \mathcal{X}} \frac{1}{2} \Vert x - y \Vert^2_2. \notag
\end{equation}

\begin{enumerate}[a)]
    \item Prove that~$\Vert \Pi_{\mathcal{X}} (x) - \Pi_{\mathcal{X}} (y) \Vert^2_2 \leqslant \Braket{\Pi_{\mathcal{X}} (x) - \Pi_{\mathcal{X}} (y), x - y}$.

    \item Prove that~$\Vert \Pi_{\mathcal{X}} (x) - \Pi_{\mathcal{X}} (y) \Vert_2 \leqslant \Vert x - y \Vert_2$.
    
    \item If we choose $\Pi_{\mathcal{X}} (y) = \mathop{\arg\min}\limits_{x \in \mathcal{X}} \Delta_{R}(x,y) $, where $\mathcal{X}$ is the $n$-dimensional simplex $\{x\in\mathbb{R}^n_{+} \vert \sum_{i=1}^n x_i=1\}$, $\Delta_{R}(x,y)$ is defined in \eqref{bregman divergence} and $R(x)=\sum_{i=1}^n x_i\log(x_i)$. Prove that $\Pi_{\mathcal{X}} (y)=\frac{y}{||y||_1}$ when $y\in\mathbb{R}^n_{++}$. (\textit{Hint: you may use the Jensen’s inequality})
\end{enumerate}


\begin{solution}
% Write your answer here.
\subsection*{(a)}
注意到:

$\braket{\Pi_{\mathcal{X}}(x)-\Pi_{\mathcal{X}}(y), x-\Pi_{\mathcal{X}}(x)} \geq 0$

$\braket{\Pi_{\mathcal{X}}(y)-\Pi_{\mathcal{X}}(x), y-\Pi_{\mathcal{X}}(y)} \geq 0 $

两式相加可得:

$\braket{\Pi_{\mathcal{X}}(x)-\Pi_{\mathcal{X}}(y), x-y-\Pi_{\mathcal{X}}(x)+\Pi_{\mathcal{X}}(y)}\geq 0$

上式可变成:

$\braket{\Pi_{\mathcal{X}}(x)-\Pi_{\mathcal{X}}(y), x-y}-\braket{\Pi_{\mathcal{X}}(x)-\Pi_{\mathcal{X}}(y), \Pi_{\mathcal{X}}(x)-\Pi_{\mathcal{X}}(y)}\geq 0$

$\therefore \Vert \Pi_{\mathcal{X}} (x) - \Pi_{\mathcal{X}} (y) \Vert^2_2 \leqslant \Braket{\Pi_{\mathcal{X}} (x) - \Pi_{\mathcal{X}} (y), x - y}$
\subsection*{(b)}
$\Braket{\Pi_{\mathcal{X}} (x)-\Pi_{\mathcal{X}} (y), x-y} 
= \Vert \Pi_{\mathcal{X}} (x)-\Pi_{\mathcal{X}} \Vert_2 \Vert x-y \Vert_2 \cos \theta$

$\geq \Vert \Pi_{\mathcal{X}} (x) - \Pi_{\mathcal{X}} (y) \Vert^2_2$

$\therefore \Vert \Pi_{\mathcal{X}} (x) - \Pi_{\mathcal{X}} (y) \Vert_2 \leqslant \Vert x - y \Vert_2 \cos \theta \leq \Vert x - y \Vert_2$
\subsection*{(c)}
$\nabla R(y)=(\log (y_1)+1, ..., log(y_n)+1)^{\top}$

$\Delta_R(x,y)=\sum^n_{i=1}x_i\log x_i -\sum^n_{i=1}y_i\log y_i- \sum^n
_{i=1}(x_i\log x_i-y_i\log y_i+x_i-y_i)$

$=\sum^n_{i=1}x_i\log x_i -\sum^n_{i=1}x_i\log y_i -\sum^n_{i=1}(x_i-y_i)$

$=\sum^n_{i=1}x_i\log \frac{x_i}{y_i}-(1-\Vert y \Vert_1)$

$\therefore \Pi_{\mathcal{X}}(y)=\arg\min\limits_{x \in \mathcal{X}}\Delta_R(x,y)=\arg\min\limits_{x \in \mathcal{X}} \sum^n_{i=1}x_i\log \frac{x_i}{y_i}$

令$f(k)=\log \frac{1}{k}$,则$f^{''}(k)=\frac{1}{k^2}>0$,$\therefore f(k)$是凸函数且非递减的

令$k_i=\frac{y_i}{x_i}$,则$\sum^n_{i=1}x_i\log \frac{x_i}{y_i}=\sum^n_{i=1}x_i f(\frac{1}{k_i})$

有Jensen's inequality得:

$\sum^n_{i=1}x_i f(\frac{1}{k_i}) \geq f(\sum^n_{i=1}\frac{x_i}{k_i})$

当且仅当$\frac{x_i}{y_i}=\frac{x_j}{y_j}(i\neq j)$时取等号

我们取$x_i=\frac{y_i}{\Vert y \Vert_1}$,发现正好满足$\frac{x_i}{y_i}=\frac{x_j}{y_j}(i\neq j)$并且$\sum_{i=1}^n x_i=1$

$\therefore x=(\frac{y_1}{\Vert y \Vert_1},...,\frac{y_n}{\Vert y \Vert_1})=\frac{y}{\Vert y \Vert_1}$

即证明了$\Pi_{\mathcal{X}}(y)=\frac{y}{\Vert y \Vert_1}$
\end{solution}


\end{document}

\documentclass[a4paper]{article}
% \usepackage[margin=1.25in]{geometry}
\usepackage[inner=2.0cm,outer=2.0cm,top=2.5cm,bottom=2.5cm]{geometry}
% \usepackage{ctex}
\usepackage{color}
\usepackage{graphicx}
\usepackage{amssymb}
\usepackage{amsmath}
\usepackage{amsthm}
\usepackage{bm}
\usepackage{hyperref}
\usepackage{multirow}
\usepackage{mathtools}
\usepackage{enumerate}

\newcommand{\homework}[5]{
    \pagestyle{myheadings}
    \thispagestyle{plain}
    \newpage
    \setcounter{page}{1}
    \noindent
    \begin{center}
    \framebox{
        \vbox{\vspace{2mm}
        \hbox to 6.28in { {\bf Optimization Methods \hfill #2} }
        \vspace{6mm}
        \hbox to 6.28in { {\Large \hfill #1 \hfill} }
        \vspace{6mm}
        \hbox to 6.28in { {\it Instructor: {\rm #3} \hfill Name: {\rm #4}, StudentId: {\rm #5}}}
        \vspace{2mm}}
    }
    \end{center}
    % \markboth{#4 -- #1}{#4 -- #1}
    \vspace*{4mm}
}


\newenvironment{solution}
{\color{blue} \paragraph{Solution.}}
{\newline \qed}

\begin{document}
%==========================Put your name and id here==========================
\homework{Homework 3}{Fall 2022}{Lijun Zhang}{Student name}{Student id}

\paragraph{Notice}
\begin{itemize}
    \item The submission email is: \textbf{optfall2022@163.com}.
    \item Please use the provided \LaTeX{} file as a template. 
    \item If you are not familiar with \LaTeX{}, you can also use Word to generate a \textbf{PDF} file.
\end{itemize}
~\\

\paragraph{Problem 1: Negative-entropy Regularization}
~\\
~\\
Please show how to compute
\begin{equation*}
\operatorname*{argmin}_{x \in \Delta^n} \quad b^\top x + c \cdot \sum_{i=1}^n x_i \ln x_i
\end{equation*}
where $\Delta^n=\{x \; | \; \sum_{i=1}^n x_i=1,x_i\geq0,i=1,\cdots,n\}$, $b\in\mathbb{R}^n$ and $c\in\mathbb{R}$.
~\\

% \begin{solution}
% Write your answer here.
% \end{solution}




\paragraph{Problem 2: One inequality constraint}
~\\
~\\
With $c\neq 0$, express the dual problem of 
\begin{align*}
    \min \quad &c^\top x \\
    \text{s.t.} \quad &f(x)\leq 0
\end{align*}
in terms of the conjugate $f^\star$. 
~\\


% \begin{solution}
% Write your answer here.
% \end{solution}

\paragraph{Problem 3: KKT conditions}
~\\
~\\
Consider the problem
\begin{align*}
\min _{x \in \mathbb{R}^{2}} \quad &x_{1}^{2}+x_{2}^{2}\\
\text { s.t. } \quad & \left(x_{1}-1\right)^{2}+\left(x_{2}-1\right)^{2} \leq 2 \\
& \left(x_{1}-1\right)^{2}+\left(x_{2}+1\right)^{2} \leq 2
\end{align*}
where $x=\left[\begin{array}{ll}x_{1} & x_{2}\end{array}\right]^{\top} \in \mathbb{R}^{2}$. \\
\begin{enumerate}[(1)]
\item Write the Lagrangian for this problem.
\item Does strong duality hold in this problem?
\item Write the KKT conditions for this optimization problem.
\end{enumerate}
~\\

% \begin{solution}
% Write your answer here.
% \end{solution}

\paragraph{Problem 4: Equality Constrained Least-squares}
~\\
~\\
Consider the equality constrained least-squares problem
\begin{align*}
    \min \quad & \frac{1}{2}\|Ax-b\|_2^2\\
    \operatorname{s.t.} \quad & Gx=h~~
\end{align*}
where $A \in \mathbf{R}^{m\times n}$ with $\mathbf{rank}~A=n$, and $G\in\mathbf{R}^{p\times n}$ with $\mathbf{rank}~G=p$.
\begin{enumerate}[(1)]
    \item Derive the Lagrange dual problem with Lagrange multiplier vector $v$.
    \item Derive expressions for the primal solution $x^\star$ and the dual solution $v^\star$.
\end{enumerate}
~\\


% \begin{solution}
% Write your answer here.
% \end{solution}


\paragraph{Problem 5: Matrix eigenvalues}
~\\
~\\
We denote by $f(A)$ the sum of the largest $r$ eigenvalues of a symmetric matrix $A \in \mathbf{S}^n$ (with $1 \le r \le n$), i.e., 
\begin{align*}
    f(A) = \sum_{k=1}^r \lambda_k (A),
\end{align*}
where $\lambda_1 (A), \cdots, \lambda_n (A)$ are the eigenvalues of $A$ sorted in decreasing order. Show that the optimal value of the optimization problem
\begin{align*}
    \max \quad & \operatorname{tr} (AX) \\
    \operatorname{s.t.} \quad & \operatorname{tr} (X) = r, \\
    & 0 \preceq X \preceq I
\end{align*}
with variable $X \in \mathbf{S}^n$ being equal to $f(A)$.
~\\



% \begin{solution}
% Write your answer here.
% \end{solution}

\end{document}

\documentclass[a4paper]{article}
% \usepackage[margin=1.25in]{geometry}
\usepackage[inner=2.0cm,outer=2.0cm,top=2.5cm,bottom=2.5cm]{geometry}
% \usepackage{ctex}
\usepackage{color}
\usepackage{graphicx}
\usepackage{amssymb}
\usepackage{amsmath}
\usepackage{amsthm}
\usepackage{bm}
\usepackage{hyperref}
\usepackage{multirow}
\usepackage{mathtools}
\usepackage{enumerate}

\newcommand{\homework}[5]{
    \pagestyle{myheadings}
    \thispagestyle{plain}
    \newpage
    \setcounter{page}{1}
    \noindent
    \begin{center}
    \framebox{
        \vbox{\vspace{2mm}
        \hbox to 6.28in { {\bf Optimization Methods \hfill #2} }
        \vspace{6mm}
        \hbox to 6.28in { {\Large \hfill #1 \hfill} }
        \vspace{6mm}
        \hbox to 6.28in { {\it Instructor: {\rm #3} \hfill Name: {\rm #4}, StudentId: {\rm #5}}}
        \vspace{2mm}}
    }
    \end{center}
    % \markboth{#4 -- #1}{#4 -- #1}
    \vspace*{4mm}
}


\newenvironment{solution}
{\color{blue} \paragraph{Solution.}}
{\newline \qed}

\begin{document}
%==========================Put your name and id here==========================
\homework{Homework 1}{Fall 2022}{Lijun Zhang}{Student name}{Student id}

\paragraph{Notice}
\begin{itemize}
    \item The submission email is: \textbf{optfall2022@163.com}.
    \item Please use the provided \LaTeX{} file as a template. 
    \item If you are not familiar with \LaTeX{}, you can also use Word to generate a \textbf{PDF} file.
\end{itemize}

\paragraph{Problem 1: Inequalities}
~\\

\noindent
Let $x\in\mathbb{R}^n,y\in\mathbb{R}^n$, where $n$ is a positive integer. Let $\|\cdot\|$ denote the Euclidean norm.
\begin{enumerate}[a)]
	\item Prove the triangle inequality $\|x+y\|\leq\|x\|+\|y\|$.
	\item Prove $\|x+y\|^2\leq(1+\epsilon)\|x\|^2+(1+\frac{1}{\epsilon})\|y\|^2$ for any $\epsilon>0$.
\end{enumerate}
\emph{Hint}: You may need the Young's inequality for products, i.e. if $a$ and $b$ are nonnegative real numbers and $p$ and $q$ are real numbers greater than 1 such that $1/p+1/q=1$, then $ab\leq\frac{a^p}{p}+\frac{b^q}{q}$.


% \begin{solution}
% Write your answer here.
% \end{solution}




\paragraph{Problem 2: Definition of convexity}
~\\

\noindent
Convex $C_c$ sets are the sets satisfying the constraints below:

$$\theta x_1 + (1-\theta)x_2 \in C_c$$
$$\text{for all, } x_1,x_2\in C_c, 0\leq \theta \leq 1$$

\noindent
Determine if each set below is convex.\\

\noindent
a) $\{(x,y)\in \mathbb{R}^2_{++}|x/y\leq 1\}$\\

\noindent
b) $\{(x,y)\in \mathbb{R}^2_{++}|x/y\geq 1\}$\\

\noindent
c) $\{(x,y)\in \mathbb{R}^2_{++}|xy\leq 1\}$\\

\noindent
d) $\{(x,y)\in \mathbb{R}^2_{++}|xy\geq 1\}$\\

\noindent
e) $\{(x,y)\in \mathbb{R}^2_{++}|y= \text{tanh}(x)=\frac{e^x-e^{-x}}{e^x+e^{-x}}\}$\\


% \begin{solution}
% Write your answer here.
% \end{solution}

\paragraph{Problem 3: Convex sets}
~\\

\begin{enumerate}[a)]
	\item Show that a polyhedron $P=\{x\in \mathbb{R}^n: Ax \leq b, A\in \mathbb{R}^{m\times n}, b\in\mathbb{R}^m\}$ is convex.
	
	\item Show that if $S\subseteq  \mathbb{R}^n$ is convex, and $A \in \mathbb{R}^{m \times n}$,
	then $A(S) = \{ Ax : x \in S \}$, is convex.
	
	\item Show that if $S\subseteq  \mathbb{R}^m$ is convex, and $A \in \mathbb{R}^{m \times n}$,
	then $A^{-1}(S) = \{ x : Ax \in S \}$, is convex.
\end{enumerate}


% \begin{solution}
% Write your answer here.
% \end{solution}

\paragraph{Problem 4: Convex cone}
~\\

\noindent
Let $K$ be a convex cone. The set $K^* = \left\{y|x^\top y \geq 0, \forall x\in K\right\}$ is called the dual cone of $K$.


\begin{enumerate}[a)]
	\item Show that $K^*$ is a convex cone (even $K$ is not convex).
	
	\item Show that a dual cone of a subspace $V \subset \mathbb{R}^n$ (which is a cone) is its orthogonal complement $V^+ = \{ y| y^\top v = 0, \forall v \in V\}$.
	
	\item What is the dual cone of the nonnegative orthant $(\mathbb{R}_+^n)$?
\end{enumerate}



% \begin{solution}
% Write your answer here.
% \end{solution}


\paragraph{Problem 5: Generalized Inequalities}
~\\

\noindent
Let~$K^*$~be the dual cone of a convex cone $K$. Prove the following,
\begin{enumerate}[a)]
    \item $K^*$~is indeed a convex cone.

    \item $K_1 \subseteq K_2$~implies~$K^*_2 \subseteq K^*_1$.
\end{enumerate}



% \begin{solution}
% Write your answer here.
% \end{solution}

\end{document}
